\documentclass[../notes.tex]{subfiles}

\pagestyle{main}
\renewcommand{\chaptermark}[1]{\markboth{\chaptername\ \thechapter\ (#1)}{}}
\setcounter{chapter}{2}

\begin{document}




\chapter{Applications of Representation Theory}
\section{Reducible Representations and Direct Products}
\begin{itemize}
    \item \marginnote{10/10:}PSet is due at the beginning of next class. Email or submit in paper. Show your work!
    \item Since molecules have more than one point, we need to work with the characters of reducible representations. In particular, when applying group theory to chemical problems, we need to find the IRRs whose sum is the reducible representation.
    \item \textbf{Reduction formula}: The formula which takes a $\Gamma_\text{red}$ and decomposes it into a sum of $\Gamma_\text{IRR}$s. \emph{Given by}
    \begin{equation*}
        n(\Gamma_A) = \frac{1}{h}\sum_ig(R)\chi_\text{IRR}(R)\chi_\text{RR}(R)
    \end{equation*}
    \begin{itemize}
        \item $n(\Gamma_A)$ is the number of times the IRR $A$ occurs in $\Gamma_\text{red}$.
        \item $h$ is the order of the group.
        \item $g(R)$ is the order of the class under the symmetry operation $R$.
        \item $\chi_\text{IRR}(R)$ is the character of the IRR under the symmetry operation $R$.
        \item $\chi_\text{RR}(R)$ is the character of the reducible representation under the symmetry operation $R$.
    \end{itemize}
    \item Recall $\Gamma_\text{red}=(3,0,1)$ in $C_{3v}$ from Lecture 2.2.
    \begin{itemize}
        \item The number of times each IRR appears is, according to the reduction formula:
        \begin{alignat*}{4}
            n(A_1) &= \frac{1}{6}(1\cdot 1\cdot 3+2\cdot{}&1&\cdot 0+3\cdot{}&1&\cdot 1) &&= 1\\
            n(A_2) &= \frac{1}{6}(1\cdot 1\cdot 3+2\cdot{}&1&\cdot 0+3\cdot{}&-1&\cdot 1) &&= 0\\
            n(E) &= \frac{1}{6}(1\cdot 2\cdot 3+2\cdot{}&-1&\cdot 0+3\cdot{}&0&\cdot 1) &&= 1
        \end{alignat*}
    \end{itemize}
    \item Simple cases (like this and the next one) we can often do by inspection.
    \item Example: $\Gamma_\text{RR}=(6,0,0)$.
    \begin{itemize}
        \item Decompose it into $A_1+A_2+2E$.
    \end{itemize}
    \item \textbf{Direct product} (of two representations): The (reducible or irreducible) representation obtained by multiplying the characters of the two representations which correspond under each operation. \emph{Denoted by} $\bm{M\times N}$, where $M,N$ are Mulliken symbols.
    \item Examples:
    \begin{itemize}
        \item $A_1\times E=E$.
        \item $A_2\times E=E$.
        \item $E\times E=A_1+A_2+E$.
        \item $A_2\times A_2=A_1$.
    \end{itemize}
    \item We now dive into how to use the symmetry properties of a collection of orbitals to determine the states that arise by populating them with electrons. To do this for a given basis set of valence atomic orbitals, we need to ask what set of IRRs they fall into.
    \item Procedure.
    \begin{enumerate}
        \item Generate the characters of this representation by examining the trace of the relevant transform matricies.
        \begin{itemize}
            \item $+1$ on the diagonal if a particular basis function is left unchanged during the symmetry operation.
            \item $0$ on the diagonal if the basis function is transformed to another function.
            \item $-1$ on the diagonal if the function is converted into minus itself.
            \item Indeed, only basis set elements that do not move contribute to the trace (i.e., character) of the representation.
        \end{itemize}
        \item Check that the dimension is greater than the largest dimension permissible in the point group. If so, we need to reduce to IRRs. This is an important step to construct SALCs.
    \end{enumerate}
    \item Example: Consider the set of $\ce{H}_{1s}$ orbitals of \ce{NH3} as the representation.
    \begin{itemize}
        \item $\Gamma_{3\ce{H}_{1s}}=(3,0,1)$ since all 3 orbitals stay under $E$, all orbitals move under $2C_3$, and 1 orbital stays under $3\sigma_v$.
        \item Decompose into $A_1+E$.
        \item Therefore, the $1s$ orbital of \ce{H} within $C_{3v}$ of \ce{NH3} transforms in $a_1+e$ symmetry.
        \item Note that we use lowercase Mulliken symbols for atomic/molecular orbitals and vibrational modes and uppercase Mulliken symbols for electronic states.
        \item This $a_1+e$ symmetry for the $\ce{H}_{1s}$ group orbitals implies that there are 3 SALC orbitals: 1 of $a_1$ symmetry and 2 of $e$ symmetry.
    \end{itemize}
    \item Example: \ce{H2O}.
    \begin{itemize}
        \item $\Gamma=(2,0,0,2)=A_1+B_2$.
    \end{itemize}
\end{itemize}



\section{Projection Operations and SALCs}
\begin{itemize}
    \item \marginnote{10/12:}PSet 2 is posted and due 10/21.
    \begin{itemize}
        \item Go over questions on the HW in the review classes prior to exams.
    \end{itemize}
    \item \textbf{Symmetry-Adapted Linear Combination}: An orthonormal linear combination of one or more sets of orthonormal functions (which are either atomic orbitals or internal coordinates of a molecule) taken in such a way that the combinations form bases for irreducible representations of the symmetry group of the molecule. \emph{Also known as} \textbf{SALC}.
    \item Last time, we investigated the SALCs of \ce{NH3}. We found them by decomposing a reducible representation of the $\ce{H}_{1s}$ basis set to its irreducible representations and naming them via the character table.
    \item Goal: What do the SALCs of \ce{NH3} look like?
    \begin{itemize}
        \item We know that we need 3 SALCs, 1 of $a_1$ symmetry and 2 of $e$ symmetry.
        \item To achieve the goal, we apply the \textbf{projection operator} to each irreducible representation.
    \end{itemize}
    \item \textbf{Projection operator} (on an IRR): The operator defined as follows, which acts on IRRs. \emph{Denoted by} $\bm{\hat{P}}$. \emph{Given by}
    \begin{equation*}
        \hat{P}(\Gamma_i) = \frac{\ell_i}{h}\sum_R\chi_i(R)\hat{R}
    \end{equation*}
    \begin{itemize}
        \item $\ell_i$ is the dimension(ality) of the IRR.
        \item $\Gamma_i$ is the IRR.
        \item $\hat{R}$ is the symmetry operation to be applied to the basis.
        \item $\chi_i(R)$ is the character of the given symmetry operation for $\Gamma_i$.
        \item In other words, we need to evaluate what happens to the $\ce{H}_{1s}$ orbitals under each symmetry operation.
    \end{itemize}
    \item It follows that we need the character table to evaluate what happens to the $\ce{H}_{1s}$ orbitals for each symmetry operation.
    \begin{itemize}
        \item Note that for the projection operator, we do \emph{not} do this by class (i.e., we do need to apply \emph{every single} symmetry operation)\footnote{Notice that the order of the class is not present in the projection operator!}.
        \item Focus on one orbital in particular, and see to which orbital each symmetry operation takes it.
    \end{itemize}
    \item \ce{NH3} example.
    \begin{itemize}
        \item We have
        \begin{align*}
            E:x_1\mapsto x_1&&
            C_3:x_1\mapsto x_2&&
            {C_3}^2:x_1\mapsto x_3&&
            \sigma_v:x_1\mapsto x_1&&
            \sigma_v':x_1\mapsto x_3&&
            \sigma_v'':x_1\mapsto x_2
        \end{align*}
        \item We can do the same for where $x_2,x_3$ go.
        \item It follows that
        \begin{equation*}
            \hat{P}(A_1)_{x_1} = 1x_1+1x_2+1x_3+1x_1+1x_3+1x_2
            = 2(x_1+x_2+x_3)
            \approx x_1+x_2+x_3
        \end{equation*}
        \begin{itemize}
            \item This implies that under the totally symmetric representation, all orbitals are the same, as we might expect. Note that the constant factor of 2 does not affect the functional form and therefore does not affect the symmetry properties.
            \item We don't carry through $\ell_i/h$ because it's a constant??
            \item Note that
            \begin{equation*}
                \hat{P}(A_1)_{x_2} = \hat{P}(A_1)_{x_3} = \hat{P}(A_1)_{x_1}
            \end{equation*}
        \end{itemize}
        \item As another example,
        \begin{equation*}
            \hat{P}(E)_{x_1} = 2x_1-x_2-x_3+0x_1+0x_3+0x_2
            = 2x_1-x_2-x_3
        \end{equation*}
        \item Similarly,
        \begin{align*}
            \hat{P}(E)_{x_2} &= 2x_2-x_3-x_1\\
            \hat{P}(E)_{x_3} &= 2x_3-x_1-x_2
        \end{align*}
        \item Note that all of these functions must be orthonormal.
        \begin{itemize}
            \item We have three functions, but we only need 2 in $e$! Thus, we employ the orthonormal rule to figure it out.
            \item We have that the first and second, and first and third are not orthogonal:
            \begin{align*}
                (\Psi_{E,x_1},\Psi_{E,x_2}) &= (2)(-1)+(2)(-1)+(-1)(-1) = 3 \neq 0\\
                (\Psi_{E,x_1},\Psi_{E,x_3}) &= (2)(-1)+(-1)(-1)+(-1)(2) = 3 \neq 0
            \end{align*}
            \item What we can do is take a linear combination of 2 and 3 so that it's orthogonal to 1.
            \item Let's try
            \begin{equation*}
                \Psi_{E,\text{new}} = \Psi_{E,x_2}-\Psi_{E,x_3}
                = 3x_2-3x_3
                \approx x_2-x_3
            \end{equation*}
            \item Indeed,
            \begin{equation*}
                (\Psi_{E,x_1},\Psi_{E,\text{new}}) = (2)(0)+(-1)(1)+(-1)(-1) = 0
            \end{equation*}
            as desired.
            \item Note that adding does not get us something orthogonal.
        \end{itemize}
        \item Normalize the SALCs.
        \begin{itemize}
            \item At this point, we have that
            \begin{align*}
                \Psi_{A_1\text{ SALC}} &= x_1+x_2+x_3&
                \Psi_{E\text{ SALC}} &= 2x_1-x_2-x_3&
                \Psi_{E\text{ SALC}} &= x_2-x_3
            \end{align*}
            \item Normalization means adjusting the normalization constant $N$ such that
            \begin{equation*}
                \int[N(x_1+x_2+x_3)]^2 = 1
            \end{equation*}
            \item More simply, we multiply each of the above by 1 over the square root of the sum of the squares of the extant coefficients. So
            \begin{align*}
                \Psi_{A_1\text{ SALC}} &= \frac{1}{\sqrt{1^2+1^2+1^2}}(x_1+x_2+x_3)&
                \Psi_{E\text{ SALC}} &= \frac{1}{\sqrt{2^2+(-1)^2+(-1)^2}}(2x_1-x_2-x_3)\\
                &= \frac{1}{\sqrt{3}}(x_1+x_2+x_3)&
                &= \frac{1}{\sqrt{6}}(2x_1-x_2-x_3)
            \end{align*}
            \begin{align*}
                \Psi_{E\text{ SALC}} &= \frac{1}{\sqrt{0^2+1^2+(-1)^2}}(x_2-x_3)\\
                &= \frac{1}{\sqrt{2}}(x_2-x_3)
            \end{align*}
        \end{itemize}
        \item Lastly, we can draw the orbitals.
        \begin{figure}[h!]
            \centering
            \begin{subfigure}[b]{0.25\linewidth}
                \centering
                \begin{tikzpicture}
                    \draw
                        (0,0) -- (1,-0.9) coordinate (x1)
                        (0,0) -- (0,-1.2) coordinate (x2)
                        (0,0) -- (-1,-0.9) coordinate (x3)
                    ;
        
                    \filldraw [semithick,fill=grz] (x1) circle (3mm);
                    \filldraw [semithick,fill=grz] (x2) circle (3mm);
                    \filldraw [semithick,fill=grz] (x3) circle (3mm);
                \end{tikzpicture}
                \caption{$\Psi_{a_1}$.}
                \label{fig:NH3SALCa}
            \end{subfigure}
            \begin{subfigure}[b]{0.25\linewidth}
                \centering
                \begin{tikzpicture}
                    \draw
                        (0,0) -- (1,-0.9) coordinate (x1)
                        (0,0) -- (0,-1.2) coordinate (x2)
                        (0,0) -- (-1,-0.9) coordinate (x3)
                    ;
        
                    \filldraw [semithick,fill=grz]   (x1) circle (4.24mm);
                    \filldraw [semithick,fill=white] (x2) circle (2.12mm);
                    \filldraw [semithick,fill=white] (x3) circle (2.12mm);
                \end{tikzpicture}
                \caption{$\Psi_{e_a}$.}
                \label{fig:NH3SALCb}
            \end{subfigure}
            \begin{subfigure}[b]{0.25\linewidth}
                \centering
                \begin{tikzpicture}
                    \draw
                        (0,0) -- (1,-0.9) coordinate (x1)
                        (0,0) -- (0,-1.2) coordinate (x2)
                        (0,0) -- (-1,-0.9) coordinate (x3)
                    ;
        
                    \filldraw [semithick,fill=grz]   (x2) circle (3.67mm);
                    \filldraw [semithick,fill=white] (x3) circle (3.67mm);
                \end{tikzpicture}
                \caption{$\Psi_{e_b}$.}
                \label{fig:NH3SALCc}
            \end{subfigure}
            \caption{\ce{NH3} SALCs.}
            \label{fig:NH3SALC}
        \end{figure}
    \end{itemize}
\end{itemize}



\section{Vibrational Modes and Symmetry}
\begin{itemize}
    \item \marginnote{10/14:}Purposes of the basis set $\to$ RRs $\to$ IRRs $\to$ SALCs workflow.
    \begin{enumerate}
        \item Helps us understand the symmetry properties of molecular vibrations (we are going to look at this first).
        \item Helps us understand MO diagrams (we will look at this after Exam 1).
    \end{enumerate}
    \item Symmetry properties of molecular vibrations: Any vibrational motion of a molecule can be decomposed in a combination of normal modes, and all normal modes form the basis of an irreducible representation of the point group of the molecule.
    \item How many normal modes do we have for a given molecule?
    \begin{itemize}
        \item $N$ atoms.
        \item $3N$ degrees of freedom.
        \item 3 translations.
        \item 3 rotations (2 for linear molecules).
        \item $3N-6$ (resp. $3N-5$) vibrations (i.e., normal modes).
    \end{itemize}
    \item How to determine normal modes.
    \begin{enumerate}
        \item Determine the point group.
        \item Consider the motion of atoms independently.
        \item Use the Cartesian displacement method, reduce $\Gamma_\text{Cartesian}$ to IRRs, and compare with the character table to determine which can be accounted for by translation $(x,y,z)$ and rotation $(R_x,R_y,R_z)$.
        \item Use the stretching and/or bending vectors as basis sets and use $\hat{P}$ to determine what the normal modes look like.
    \end{enumerate}
    \item Example: \ce{H2O}.
    \begin{enumerate}
        \item $C_{2v}$.
        \item 3 degrees of freedom for each atom (atom $i$ can move in the $x_i,y_i,z_i$ direction for $i=1,2,3$).
        \item Multiple steps:
        \begin{enumerate}
            \item Find $\Gamma_\text{unmoved}=(3,1,1,3)$.
            \item Find $\Gamma_{xyz}$: From the relevant character table, $\Gamma_{xyz}=A_1+B_1+B_2=(3,-1,1,1)$.
            \item Find $\Gamma_\text{Cartesian}=\Gamma_\text{unmoved}\times\Gamma_{xyz}=(9,-1,1,3)$.
            \item Apply the reduction formula: $\Gamma_\text{Cartesian}=3A_1+A_2+2B_1+3B_2$.
            \item Notice that $A_1$ corresponds to the $z$-translation, $A_2$ corresponds to the $z$-rotation, $B_1$ corresponds to the $x$-translation and $y$-rotation, and $B_2$ corresponds to the $y$-translation and $x$-rotation. If we want to determine the vibrational modes of symmetry, we need to subtract out the modes corresponding to translations and rotations of the full molecule. Thus,
            \begin{equation*}
                \Gamma_\text{vibs} = 3A_1+A_2+2B_1+3B_2-(A_1+A_2+2B_1+2B_2)
                = 2A_1+B_2
            \end{equation*}
        \end{enumerate}
        \item Multiple steps:
        \begin{enumerate}
            \item \underline{Stretching IRR(s)}: Label the bond vectors $r_1,r_2$. Find their representation.
            \begin{equation*}
                \Gamma_1 = (2,0,0,2)
                = A_1+B_2
            \end{equation*}
            \begin{itemize}
                \item Since $\Gamma_1$ decomposes into two IRRs, this basis set accounts for 2/3 of the normal vibrational modes.
                \item To get the last, we'll need another basis set, but we'll do that later.
            \end{itemize}
            \item \underline{Stretching SALC(s)}: Apply the projection operator to these normal modes.
            \begin{align*}
                \hat{P}(A_1)_{r_1} &= 2(r_1+r_2) \approx r_1+r_2\\
                \hat{P}(B_2)_{r_1} &= 2(r_1-r_2) \approx r_1-r_2
            \end{align*}
            \item \underline{Bending IRR(s)}: Label the bending basis (angle between $r_1,r_2$) $\Delta\theta$. Find its representation.
            \begin{equation*}
                \Gamma_2 = (1,1,1,1)
                = A_1
            \end{equation*}
            \item \underline{Bending SALC(s)}: Apply the projection operator to this normal mode.
            \begin{equation*}
                \hat{P}(A_1)_{\Delta\theta} = 4\Delta\theta \approx \Delta\theta
            \end{equation*}
            \item \underline{Visualize the normal modes}: $r_1+r_2$ corresponds to a symmetric stretch $\nu_s$, $r_1-r_2$ corresponds to an asymmetric stretch $\nu_a$, and $\Delta\theta$ corresponds to a bend $\delta$.
            \begin{figure}[h!]
                \centering
                \footnotesize
                \begin{subfigure}[b]{0.2\linewidth}
                    \centering
                    \chemfig[bond style={white}]{@{H1}H-[:37.75]@{O}O-[:-37.75]@{H2}H}
                    \chemmove{
                        \draw [latex-,shorten <=2pt,shorten >=2pt] (H1) -- (O);
                        \draw [latex-,shorten <=2pt,shorten >=2pt] (H2) -- (O);
                    }
                    \caption{$a_1$ stretch: $\nu_s$.}
                    \label{fig:H2Ovibsa}
                \end{subfigure}
                \begin{subfigure}[b]{0.2\linewidth}
                    \centering
                    \chemfig[bond style={white}]{@{H1}H-[:37.75]@{O}O-[:-37.75]@{H2}H}
                    \chemmove{
                        \draw [latex-,shorten <=2pt,shorten >=2pt] (H1) -- (O);
                        \draw [-latex,shorten <=2pt,shorten >=2pt] (H2) -- (O);
                    }
                    \caption{$b_2$ stretch: $\nu_a$.}
                    \label{fig:H2Ovibsb}
                \end{subfigure}
                \begin{subfigure}[b]{0.2\linewidth}
                    \centering
                    \chemfig{@{H1}H-[:37.75]@{O}O-[:-37.75]@{H2}H}
                    \chemmove{
                        \pic [draw,{latex[flex]}-{latex[flex]},shorten <=2pt,shorten >=2pt] {angle=H1--O--H2};
                    }
                    \caption{$a_1$ stretch: $\delta$.}
                    \label{fig:H2Ovibsc}
                \end{subfigure}
                \caption{\ce{H2O} vibrational modes.}
                \label{fig:H2Ovibs}
            \end{figure}
            \item \underline{Quantum mechanically calculate the stretching frequencies}: For \ce{H2O}, $\nu_s=\SI{3657}{\per\centi\meter}$, $\nu_a=\SI{3756}{\per\centi\meter}$, and $\delta=\SI{1595}{\per\centi\meter}$.
        \end{enumerate}
    \end{enumerate}
    \item Example: \ce{PH3} stretching modes.
    \begin{itemize}
        \item $C_{3v}$.
        \item $\Gamma_\nu=(3,0,1)=a_1+e$.
        \item Projecting:
        \begin{align*}
            \hat{P}(A_1)_{r_1} &\approx r_1+r_2+r_3&
            \hat{P}(E)_{r_1} &\approx 2r_1-r_2-r_3&
            \hat{P}(E)_{r_2-r_3} &\approx r_2-r_3
        \end{align*}
        \item Drawing:
        \begin{figure}[h!]
            \centering
            \footnotesize
            \begin{subfigure}[b]{0.2\linewidth}
                \centering
                \chemfig[bond style={white}]{@{P}P(-[:-30]@{H1}H)(<[:-110]@{H2}H)(<:[:-150]@{H3}H)}
                \chemmove{
                    \draw [latex-,shorten <=2pt,shorten >=2pt] (H1) -- (P);
                    \draw [latex-,shorten <=2pt,shorten >=2pt] (H2) -- (P);
                    \draw [latex-,shorten <=2pt,shorten >=2pt] (H3) -- (P);
                }
                \caption{$a_1$ stretch.}
                \label{fig:PH3vibsa}
            \end{subfigure}
            \begin{subfigure}[b]{0.2\linewidth}
                \centering
                \chemfig[bond style={white}]{@{P}P(-[:-30,2]@{H1}H)(<[:-110]@{H2}H)(<:[:-150]@{H3}H)}
                \chemmove{
                    \draw [latex-,shorten <=2pt,shorten >=2pt] (H1) -- (P);
                    \draw [-latex,shorten <=2pt,shorten >=2pt] (H2) -- (P);
                    \draw [-latex,shorten <=2pt,shorten >=2pt] (H3) -- (P);
                }
                \caption{$e$ stretch (1).}
                \label{fig:PH3vibsb}
            \end{subfigure}
            \begin{subfigure}[b]{0.2\linewidth}
                \centering
                \chemfig[bond style={white}]{@{P}P(-[:-30,,,,black]@{H1}H)(<[:-110]@{H2}H)(<:[:-150]@{H3}H)}
                \chemmove{
                    \draw [-latex,shorten <=2pt,shorten >=2pt] (H2) -- (P);
                    \draw [latex-,shorten <=2pt,shorten >=2pt] (H3) -- (P);
                }
                \caption{$e$ stretch (2).}
                \label{fig:PH3vibsc}
            \end{subfigure}
            \caption{\ce{PH3} vibrational modes.}
            \label{fig:PH3vibs}
        \end{figure}
    \end{itemize}
    \item Note that we don't need $\Gamma_{3N}$ to derive $\Gamma_\nu$! We would only need it for $\Gamma_\delta$, unless Wuttig gives us a bending basis with which to work.
    \item Some observation on orthogonal projections.
    \begin{itemize}
        \item Suppose we want to derive $3r_2-3r_3$. Since $\hat{P}$ is a linear operator, we can equally well take the difference $\hat{P}(E)_{r_2}-\hat{P}(E)_{r_3}$ and project out $r_2-r_3$ via $\hat{P}_{r_2-r_3}$ to start.
        \item Moreover, I suspect that the projection operator is unitary (i.e., maps orthogonal vectors to orthogonal vectors). At least in this case, notice that $r_1$ and $r_2-r_3$ are very much orthogonal (see Figure \ref{fig:orthoStretch}), just like their projections.
        \begin{figure}[h!]
            \centering
            \footnotesize
            \begin{tikzpicture}
                \draw [-latex] (0,0) coordinate (O) -- (90:1) coordinate (1) node[above]{$r_1$};
                \draw [-latex] (0,0) -- (-30:1) node[below right=-1.5pt]{$r_2$};
                \draw [-latex] (0,0) -- (-150:1) node[below left=-1.5pt]{$r_3$};
                \draw [-latex] (-30:1) -- node[near end,below]{$-r_3$} ++(30:1) coordinate (A);
                \draw [-latex] (0,0) -- node[pos=0.6,above]{$r_2-r_3$} (A);
        
                \pic [draw,angle radius=2mm] {right angle=A--O--1};
            \end{tikzpicture}
            \caption{Orthogonal stretching basis.}
            \label{fig:orthoStretch}
        \end{figure}
    \end{itemize}
\end{itemize}




\end{document}