\documentclass[../notes.tex]{subfiles}

\pagestyle{main}
\renewcommand{\chaptermark}[1]{\markboth{\chaptername\ \thechapter\ (#1)}{}}
\setcounter{chapter}{4}

\begin{document}




\chapter{Exam and Intro to MO Theory}
\section{First Exam Review}
\begin{itemize}
    \item \marginnote{10/17:}Arrive by 9:25 AM on Wednesday.
    \item Probably four problems.
    \begin{itemize}
        \item Won't be doing IRRs for $T_d$ point groups or anything really complicated like that.
    \end{itemize}
    \item All character tables we need to solve the questions will be provided.
    \item The questions are independent; if we can't solve problem 1, we'll still be able to take a shot at problem 2.
    \item Question: Assigning Cartesian coordinates or $R_{xyz}$ to the Mulliken symbols.
    \begin{itemize}
        \item Example: Assign $R_z$ and $x^2-y^2$ in $D_4$.
        \item Draw the $d_{x^2-y^2}$ orbital and $R_z$ vector in 3D space.
        \item Subject the drawings to each symmetry operation.
        \begin{itemize}
            \item For $E,C_4,C_2,C_2',C_2''$, $d_{x^2-y^2}$ gets sent to itself, it's inverse, itself, itself, and it's inverse. Thus, $\Gamma=(1,-1,1,1,-1)=B_1$.
            \item Likewise, we have for $R_z$ that $\Gamma=(1,1,1,-1,-1)=A_2$.
            \item $\Gamma_{p_y}=(1,0,\dots)$. Once you get to $\chi(C_4)=0$, you know that it must be $E$. For $C_2'$, $p_x\mapsto 1$ and $p_y\mapsto -1$ or vice versa, so we sum these two to get 0. Since $p_x,p_y$ both have $\chi(C_4)=0$, they must be degenerate and equal to $E$.
        \end{itemize}
        \item On Wednesday, you don't need to draw out every image, but you need some justification, e.g., "I did a \ang{90} turn."
    \end{itemize}
    \item Memorize Mulliken symbol rules!
    \item Question: Problem C on PSet 2.
    \begin{itemize}
        \item Ignore all of the text about qubits.
        \item First step: Figure out what the basis is (it's the outer plane vectors, i.e., the arrows).
        \item Point group: $D_{4h}$.
        \item We need to find $\Gamma_\text{out-of-plane}$, a reducible representation! We're finding a RR for all of the arrows at once (not projecting one into the others), and then decomposing it into IRRs and hoping that one of them is $a_{2u}$.
        \item We get
        \begin{center}
            \small
            \renewcommand{\arraystretch}{1.2}
            \begin{tabular}{c|cccccccccc}
                $D_{4h}$ & $E$ & $2C_4$ & $C_2$ & $2C_2'$ & $2C_2''$ & $i$ & $2S_1$ & $\sigma_h$ & $2\sigma_v$ & $2\sigma_d$\\
                \hline
                $\Gamma$ & 4 & 0 & 0 & $-2$ & 0 & 0 & 0 & $-4$ & 2 & 0
            \end{tabular}
        \end{center}
        \item We can reduce this down to
        \begin{equation*}
            \Gamma = a_{2u}+b_{2u}+e_g
        \end{equation*}
        to identify $a_{2u}$.
        \item Similarly, in $C_{4v}$, we can reduce $\Gamma=(4,0,0,2,0)=a_1+b_1+e$ to identify $a_1$.
    \end{itemize}
    \item Review PSets and answers! I keep missing easy things. Really break down a procedure to attack the different types of problems.
    \item Question: Assign symmetries based off of IR spectra.
    \begin{itemize}
        \item Example: Vibrational spectroscopy has played a role in supporting the structure of the ion shown below. Raman spectroscopy of the tetramethylammonium salt of this ion shows a single absorption in the region expected for \ce{I=O} stretching vibrations at \SI{789}{\per\centi\meter}. Is a single Raman band consistent with the proposed trans orientation of the oxygen atoms? Rationalize your answer.
        \item Approach: In $D_{5h}$, create an IRR with the two \ce{I=O} bond vectors as your basis. Thus,
        \begin{center}
            \small
            \renewcommand{\arraystretch}{1.2}
            \begin{tabular}{c|cccccccc}
                $D_{5h}$ & $E$ & $2C_5$ & $2{C_5}^2$ & $5C_2'$ & $\sigma_h$ & $2S_5$ & $2{S_5}^3$ & $5\sigma_v$\\
                \hline
                $\Gamma$ & 2 & 2 & 2 & 0 & 0 & 0 & 0 & 2
            \end{tabular}
        \end{center}
        \item We have that $\Gamma=a_1'+a_2''$. Since only one of these ($a_1'$) is Raman active, a single Raman band is consistent with this structure.
    \end{itemize}
    \item Question: Finding vibrational modes.
    \begin{itemize}
        \item Example: What are the normal modes of \emph{trans}-\ce{N2F2}.
        \item Not linear, so we expect $3N-6=6$ normal modes to deal with.
        \item We begin doing the Cartesian displacement method.
        \begin{center}
            \small
            \renewcommand{\arraystretch}{1.2}
            \begin{tabular}{c|cccc}
                $C_{2h}$ & $E$ & $C_2(z)$ & $i$ & $\sigma_h$\\
                \hline
                $\Gamma_\text{unm}$ & 4 & 0 & 0 & 4\\
                $\Gamma_{xyz}$ & 3 & -- & -- & 1\\
                \hline
                $\Gamma_{3N}$ & 12 & 0 & 0 & 4\\
            \end{tabular}
        \end{center}
        \begin{itemize}
            \item Note that we use dashes in two entries of $\Gamma_{xyz}$ because since there are zeroes above and we are multiplying, it does not matter what these values are (they will end up being zero in the direct product, regardless).
        \end{itemize}
        \item We can reduce to $\Gamma_{3N}=4A_g+2B_g+2A_u+4B_u$. Subtracting out $\Gamma_\text{rot}=A_g+2B_g$ and $\Gamma_{xyz}=A_u+2B_u$, we end up with $\Gamma_\text{vib}=3A_g+A_u+2B_u$.
        \item Let's go in depth. The spectroscopic activities are:
        \begin{itemize}
            \item $A_u+2B_u$ are IR active.
            \item $3A_g$ is Raman active.
        \end{itemize}
        \item Let's go even more in depth.
        \begin{itemize}
            \item $A_u$ is the only vibration that has $\chi(\sigma_h)=-1$, i.e., it is the only one that vibrates out-of-plane.
            \item If $\chi(\sigma_h)=+1$, then the vibration is symmetric with respect to $\sigma_h$. This must mean that the molecules are entirely confined to the plane. If $\chi(\sigma_h)=-1$, then we have an out of plane vibration (e.g., \ce{F} atoms going above and below the plane in equal and opposite amounts --- there is a motion that can be inverted). If one \ce{F} atom goes twice as far, this is probably $E$ (think about what would be required for the projection operator).
            \item If you get pictures of normal modes (hint hint!!), you can retroactively get Mulliken symbols by observing the symmetry with respect to the observations.
        \end{itemize}
    \end{itemize}
\end{itemize}



\section{Midterm Thoughts}
\begin{itemize}
    \item Generating a RR based on a basis set tells us the symmetry of something (vibrations, stretches, orbitals, ...). Projecting out an IRR on a basis gives us a specific example of something (an actual molecular orbital, vibrational mode, ...).
    \item Determining the depolarization ratio. Determine which Mulliken symbols are the \emph{most} symmetric, i.e., symmetric to the most elements. For example, $A_2$ is more symmetric than $B_2$ because it is antisymmetric to 1 element instead of 2 elements.
\end{itemize}




\end{document}