\documentclass[../notes.tex]{subfiles}

\pagestyle{main}
\renewcommand{\chaptermark}[1]{\markboth{\chaptername\ \thechapter\ (#1)}{}}
\setcounter{chapter}{4}

\begin{document}




\chapter{Exam and Intro to MO Theory}
\section{First Exam Review}
\begin{itemize}
    \item \marginnote{10/17:}Arrive by 9:25 AM on Wednesday.
    \item Probably four problems.
    \begin{itemize}
        \item Won't be doing IRRs for $T_d$ point groups or anything really complicated like that.
    \end{itemize}
    \item All character tables we need to solve the questions will be provided.
    \item The questions are independent; if we can't solve problem 1, we'll still be able to take a shot at problem 2.
    \item Question: Assigning Cartesian coordinates or $R_{xyz}$ to the Mulliken symbols.
    \begin{itemize}
        \item Example: Assign $R_z$ and $x^2-y^2$ in $D_4$.
        \item Draw the $d_{x^2-y^2}$ orbital and $R_z$ vector in 3D space.
        \item Subject the drawings to each symmetry operation.
        \begin{itemize}
            \item For $E,C_4,C_2,C_2',C_2''$, $d_{x^2-y^2}$ gets sent to itself, it's inverse, itself, itself, and it's inverse. Thus, $\Gamma=(1,-1,1,1,-1)=B_1$.
            \item Likewise, we have for $R_z$ that $\Gamma=(1,1,1,-1,-1)=A_2$.
            \item $\Gamma_{p_y}=(1,0,\dots)$. Once you get to $\chi(C_4)=0$, you know that it must be $E$. For $C_2'$, $p_x\mapsto 1$ and $p_y\mapsto -1$ or vice versa, so we sum these two to get 0. Since $p_x,p_y$ both have $\chi(C_4)=0$, they must be degenerate and equal to $E$.
        \end{itemize}
        \item On Wednesday, you don't need to draw out every image, but you need some justification, e.g., "I did a \ang{90} turn."
    \end{itemize}
    \item Memorize Mulliken symbol rules!
    \item Question: Problem C on PSet 2.
    \begin{itemize}
        \item Ignore all of the text about qubits.
        \item First step: Figure out what the basis is (it's the outer plane vectors, i.e., the arrows).
        \item Point group: $D_{4h}$.
        \item We need to find $\Gamma_\text{out-of-plane}$, a reducible representation! We're finding a RR for all of the arrows at once (not projecting one into the others), and then decomposing it into IRRs and hoping that one of them is $a_{2u}$.
        \item We get
        \begin{center}
            \small
            \renewcommand{\arraystretch}{1.2}
            \begin{tabular}{c|cccccccccc}
                $D_{4h}$ & $E$ & $2C_4$ & $C_2$ & $2C_2'$ & $2C_2''$ & $i$ & $2S_1$ & $\sigma_h$ & $2\sigma_v$ & $2\sigma_d$\\
                \hline
                $\Gamma$ & 4 & 0 & 0 & $-2$ & 0 & 0 & 0 & $-4$ & 2 & 0
            \end{tabular}
        \end{center}
        \item We can reduce this down to
        \begin{equation*}
            \Gamma = a_{2u}+b_{2u}+e_g
        \end{equation*}
        to identify $a_{2u}$.
        \item Similarly, in $C_{4v}$, we can reduce $\Gamma=(4,0,0,2,0)=a_1+b_1+e$ to identify $a_1$.
    \end{itemize}
    \item Review PSets and answers! I keep missing easy things. Really break down a procedure to attack the different types of problems.
    \item Question: Assign symmetries based off of IR spectra.
    \begin{itemize}
        \item Example: Vibrational spectroscopy has played a role in supporting the structure of the ion shown below. Raman spectroscopy of the tetramethylammonium salt of this ion shows a single absorption in the region expected for \ce{I=O} stretching vibrations at \SI{789}{\per\centi\meter}. Is a single Raman band consistent with the proposed trans orientation of the oxygen atoms? Rationalize your answer.
        \item Approach: In $D_{5h}$, create an IRR with the two \ce{I=O} bond vectors as your basis. Thus,
        \begin{center}
            \small
            \renewcommand{\arraystretch}{1.2}
            \begin{tabular}{c|cccccccc}
                $D_{5h}$ & $E$ & $2C_5$ & $2{C_5}^2$ & $5C_2'$ & $\sigma_h$ & $2S_5$ & $2{S_5}^3$ & $5\sigma_v$\\
                \hline
                $\Gamma$ & 2 & 2 & 2 & 0 & 0 & 0 & 0 & 2
            \end{tabular}
        \end{center}
        \item We have that $\Gamma=a_1'+a_2''$. Since only one of these ($a_1'$) is Raman active, a single Raman band is consistent with this structure.
    \end{itemize}
    \item Question: Finding vibrational modes.
    \begin{itemize}
        \item Example: What are the normal modes of \emph{trans}-\ce{N2F2}.
        \item Not linear, so we expect $3N-6=6$ normal modes to deal with.
        \item We begin doing the Cartesian displacement method.
        \begin{center}
            \small
            \renewcommand{\arraystretch}{1.2}
            \begin{tabular}{c|cccc}
                $C_{2h}$ & $E$ & $C_2(z)$ & $i$ & $\sigma_h$\\
                \hline
                $\Gamma_\text{unm}$ & 4 & 0 & 0 & 4\\
                $\Gamma_{xyz}$ & 3 & -- & -- & 1\\
                \hline
                $\Gamma_{3N}$ & 12 & 0 & 0 & 4\\
            \end{tabular}
        \end{center}
        \begin{itemize}
            \item Note that we use dashes in two entries of $\Gamma_{xyz}$ because since there are zeroes above and we are multiplying, it does not matter what these values are (they will end up being zero in the direct product, regardless).
        \end{itemize}
        \item We can reduce to $\Gamma_{3N}=4A_g+2B_g+2A_u+4B_u$. Subtracting out $\Gamma_\text{rot}=A_g+2B_g$ and $\Gamma_{xyz}=A_u+2B_u$, we end up with $\Gamma_\text{vib}=3A_g+A_u+2B_u$.
        \item Let's go in depth. The spectroscopic activities are:
        \begin{itemize}
            \item $A_u+2B_u$ are IR active.
            \item $3A_g$ is Raman active.
        \end{itemize}
        \item Let's go even more in depth.
        \begin{itemize}
            \item $A_u$ is the only vibration that has $\chi(\sigma_h)=-1$, i.e., it is the only one that vibrates out-of-plane.
            \item If $\chi(\sigma_h)=+1$, then the vibration is symmetric with respect to $\sigma_h$. This must mean that the molecules are entirely confined to the plane. If $\chi(\sigma_h)=-1$, then we have an out of plane vibration (e.g., \ce{F} atoms going above and below the plane in equal and opposite amounts --- there is a motion that can be inverted). If one \ce{F} atom goes twice as far, this is probably $E$ (think about what would be required for the projection operator).
            \item If you get pictures of normal modes (hint hint!!), you can retroactively get Mulliken symbols by observing the symmetry with respect to the observations.
        \end{itemize}
    \end{itemize}
\end{itemize}



\section{Midterm Thoughts}
\begin{itemize}
    \item \marginnote{10/26:}Generating a RR based on a basis set tells us the symmetry of something (vibrations, stretches, orbitals, ...). Projecting out an IRR on a basis gives us a specific example of something (an actual molecular orbital, vibrational mode, ...).
    \item Determining the depolarization ratio. Determine which Mulliken symbols are the \emph{most} symmetric, i.e., symmetric to the most elements. For example, $A_2$ is more symmetric than $B_2$ because it is antisymmetric to 1 element instead of 2 elements.
\end{itemize}



\section{MO Theory: Fundamental Concepts and Diatomics}
\begin{itemize}
    \item \marginnote{10/28:}The name of the game for the second half of the course is \emph{electronic} spectroscopy.
    \begin{itemize}
        \item The first half was \emph{vibrational} spectroscopy.
    \end{itemize}
    \item \textbf{Molecular orbital theory}: A theory of orbitals that posits that they are in general spread over the entirety of the molecule, so considerations of molecular symmetry properties are useful in this theory. \emph{Also known as} \textbf{MO theory}.
    \item \textbf{Linear combination of atomic orbitals}: An approximation employed to reduce the notion of an MO to a practical form. \emph{Also known as} \textbf{LCAO}.
    \begin{itemize}
        \item Each MO is written as a linear combination of atomic orbitals on the various atoms.
        \item The fact that this is an \emph{approximation} is very important to note.
        \item We use a simple basis set (e.g., the atomic orbitals) and let our molecular orbital be a linear combination of them.
        \item For \ce{H2} for example, our LCAO is $\Psi=c_1\phi_1+c_2\phi_2$, where $\phi_1$ denotes the atomic orbital on \ce{H1} and $\phi_2$ denotes the atomic orbital on \ce{H2}.
    \end{itemize}
    \item We obtain an expression for the energy $E$ by applying the Hamiltonian $\hat{H}$ on our guess function (use the LCAO) and integrating over all space.
    \begin{equation*}
        \langle E\rangle = \frac{\int_{-\infty}^\infty\Psi^*\hat{H}\Psi\dd\tau}{\int_{-\infty}^\infty\Psi^*\Psi\dd\tau}
        = \frac{\int_{-\infty}^\infty(c_1\phi_1+c_2\phi_2)\hat{H}(c_1\phi_1+c_2\phi_2)\dd\tau}{\int_{-\infty}^\infty(c_1\phi_1+c_2\phi_2)\dd\tau}
    \end{equation*}
    \begin{itemize}
        \item Denote the numerator in the rightmost term above by $N$ and the corresponding denominator by $D$.
        \item Then
        \begin{equation*}
            N = c_1^2\underbrace{\int\phi_1\hat{H}\phi_1\dd\tau}_{H_{11}}+c_1c_2\underbrace{\int\phi_1\hat{H}\phi_2\dd\tau}_{H_{12}}+c_2c_1\underbrace{\int\phi_2\hat{H}\phi_1\dd\tau}_{H_{21}}+c_2^2\underbrace{\int\phi_2\hat{H}\phi_2\dd\tau}_{H_{22}}
        \end{equation*}
        \begin{itemize}
            \item $H_{11}$ is the energy of the electron in orbital 1.
            \item $H_{12}$ and $H_{21}$ are the \textbf{resonance integrals}.
            \item $H_{22}$ is the energy of the electron in orbital 2.
        \end{itemize}
        \item It follows that
        \begin{equation*}
            N = c_1^2H_{11}+c_1c_2H_{12}+c_2c_1H_{21}+c_2^2H_{22}
        \end{equation*}
        \item Additionally,
        \begin{equation*}
            D = c_1^2\underbrace{\int\phi_1\phi_1\dd\tau}_1+c_1c_2\underbrace{\int\phi_1\phi_2\dd\tau}_S+c_2c_1\underbrace{\int\phi_2\phi_1\dd\tau}_S+c_2^2\underbrace{\int\phi_2\phi_2\dd\tau}_1
        \end{equation*}
        \begin{itemize}
            \item The left- and rightmost integrals above evaluate to 1 because the AOs are normalized.
            \item The middle two integrals above are the \textbf{overlap integral}.
        \end{itemize}
        \item It follows that
        \begin{equation*}
            D = c_1^2+2c_1c_2S+c_2^2
        \end{equation*}
        \item Using our above results and the \textbf{secular determinant} (which we will not cover in this course since the focus is not on quantum mechanics), we find that
        \begin{equation*}
            E = \frac{H\pm H_{12}}{1\pm S}
        \end{equation*}
    \end{itemize}
    \item \textbf{Resonance integral}: An integral describing the strength of the bonding interaction due to the overlap of AOs in the MO.
    \item \textbf{Overlap integral}: An integral that measures the effectiveness of the overlap.
    \begin{itemize}
        \item Dependent on the type of bond. In particular, it is dependent on two factors.
        \begin{enumerate}
            \item Spatial overlap ($\sigma>\pi>\delta$ bonds).
            \item Energy overlap (based on \textbf{Valence Orbital Ionization Energies} [given in electron volts]).
            \begin{itemize}
                \item As a first approximation, oribtals separated by greater than 1 Rydberg ($R_H=\SI{13.6}{\electronvolt}$) do \emph{not} overlap significantly\footnote{This is related to Talapin's arbitrary \SI{10}{\electronvolt}!}.
            \end{itemize}
        \end{enumerate}
    \end{itemize}
    \item \textbf{Valence Orbital Ionization Energy}: Exactly like it sounds. \emph{Also known as} \textbf{VOIE}.
    \item This result leads to \ce{H2}'s MOs. Ignoring overlap (i.e., taking $S=0$) we have the following orbital diagram.
    \begin{figure}[h!]
        \centering
        \footnotesize
        \begin{tikzpicture}[
            every node/.style=black
        ]
            \draw [gry,ultra thick]
                (-2,0) coordinate (H1) -- node{\Large$\upharpoonleft$} node[below=2mm]{\ce{H}} ++(-0.5,0)
                (2,0) coordinate (H2) -- node{\Large$\upharpoonleft$} node[below=2mm]{\ce{H}} ++(0.5,0)
            ;
            \draw [ultra thick]
                (-0.25,1) coordinate (Hs2l) -- ++(0.5,0) coordinate (Hs2r)
                (-0.25,-1) coordinate (Hsl) -- node{\Large$\upharpoonleft$\hspace{-1mm}$\downharpoonright$} ++(0.5,0) coordinate (Hsr)
            ;
    
            \draw [grx,thick,densely dashed]
                (H1) -- (Hsl)
                (H1) -- (Hs2l)
                (H2) -- (Hsr)
                (H2) -- (Hs2r)
                (H1) -- (H2)
            ;
    
            \draw [semithick,latex-latex,shorten <=2pt,shorten >=2pt] ($(H1)!0.5!(H2)$) -- node[right]{$H_{12}$} ($(Hs2l)!0.5!(Hs2r)$);
    
            \draw [semithick,-latex] (-3,-1.5) -- node[left]{$E$} ++(0,3);
            \path (-3.5,0) -- (3.5,0);
        \end{tikzpicture}
        \caption{\ce{H2} MO diagram.}
        \label{fig:MOsH2}
    \end{figure}
    \item \textbf{Bond order}: The following quantity. \emph{Denoted by} \textbf{B.O.}. \emph{Given by}
    \begin{equation*}
        \text{B.O.} = \frac{\text{\# electrons in bonding orbitals}-\text{\# electrons in antibonding orbitals}}{2}
    \end{equation*}
    \begin{itemize}
        \item For \ce{H2}, $\text{B.O.}=1$.
    \end{itemize}
    \item Now we do \ce{He2}.
    \begin{figure}[H]
        \centering
        \footnotesize
        \begin{tikzpicture}[
            every node/.style=black
        ]
            \draw [gry,ultra thick]
                (-2,0) coordinate (He1) -- node{\Large$\upharpoonleft$\hspace{-1mm}$\downharpoonright$} node[below=2mm]{\ce{He}} ++(-0.5,0)
                (2,0) coordinate (He2) -- node{\Large$\upharpoonleft$\hspace{-1mm}$\downharpoonright$} node[below=2mm]{\ce{He}} ++(0.5,0)
            ;
            \draw [ultra thick]
                (-0.25,1.5) coordinate (Hes2l) -- node{\Large$\upharpoonleft$\hspace{-1mm}$\downharpoonright$} ++(0.5,0) coordinate (Hes2r)
                (-0.25,-1.5) coordinate (Hesl) -- node{\Large$\upharpoonleft$\hspace{-1mm}$\downharpoonright$} ++(0.5,0) coordinate (Hesr)
            ;
            \draw [rey,ultra thick]
                (-0.25,2.2) coordinate (Hes2la) -- node{\Large$\upharpoonleft$\hspace{-1mm}$\downharpoonright$} ++(0.5,0) coordinate (Hes2ra)
                (-0.25,-0.8) coordinate (Hesla) -- node{\Large$\upharpoonleft$\hspace{-1mm}$\downharpoonright$} ++(0.5,0) coordinate (Hesra)
            ;
    
            \draw [grx,thick,densely dashed]
                (He1) -- (Hesl)
                (He1) -- (Hes2l)
                (He2) -- (Hesr)
                (He2) -- (Hes2r)
            ;
            \draw [rex,thick,densely dashed]
                (He1) -- (Hesla)
                (He1) -- (Hes2la)
                (He2) -- (Hesra)
                (He2) -- (Hes2ra)
            ;
    
            \draw [semithick,-latex] (-3,-2) -- node[left]{$E$} (-3,2.7);
            \path (-3.5,0) -- (3.5,0);
        \end{tikzpicture}
        \caption{\ce{He2} MO diagram.}
        \label{fig:MOsHe2}
    \end{figure}
    \begin{itemize}
        \item Here, we get
        \begin{align*}
            E_+ &= \frac{H+H_{12}}{1+S}&
            E_- &= \frac{H-H_{12}}{1-S}
        \end{align*}
        \item The bond order is zero.
        \item No consideration of $S$ leads to the green diagram.
        \item If we assume $S\neq 0$, then we get the red diagram. Note that the antibonding is more destabilized than the bonding is stabilized. This is key!
    \end{itemize}
    \item Investigation of the diatomics of the 2nd row main group elements.
    \begin{itemize}
        \item The dividing line between \ce{N2} and \ce{O2}: \ce{O2} and heavier have the "normal" filling order, while \ce{N2} and below have significant mixing.
        \item When mixing is significant, $\sigma,\sigma^*$ for the 2s orbitals both go down in energy, i.e., become more bonding; $\sigma,\sigma^*$ for the 2p orbitals both go up in energy, i.e., become more antibonding.
    \end{itemize}
    \item Fundamentals when generating molecular orbitals.
    \begin{enumerate}
        \item When generating molecular orbitals, starting with a given number of atomic orbitals, we must generate the same number of molecular orbitals.
        \item Molecular orbitals must have the same symmetries as the atomic orbitals of which they are composed.
        \item Molecular orbitals should be within 1 Rydberg to mix.
    \end{enumerate}
    \item Strategy for MOs of heterodiatomics:
    \begin{enumerate}
        \item Determine VOIEs.
        \item Determine symmetry equivalence.
        \item Draw correlations.
        \item Look for potential $s$-$p$-$d$ mixing.
        \item Fill electrons, determine BO.
    \end{enumerate}
    \item \ce{HF} example.
    \begin{itemize}
        \item See Figure III.10 and the associated discussion from \textcite{bib:CHEM20100Notes}.
        \item No consideration of point groups necessary.
        \item We know we can only mix in $\sigma$ interactions (i.e., s with 2pz), not $\pi$ interactions with px or py.
    \end{itemize}
    \item Review QMech notes and come up with questions!
\end{itemize}




\end{document}