\documentclass[../notes.tex]{subfiles}

\pagestyle{main}
\renewcommand{\chaptermark}[1]{\markboth{\chaptername\ \thechapter\ (#1)}{}}
\setcounter{chapter}{8}

\begin{document}




\chapter{Selection Rules and Exam}
\section{Electronic Spectroscopy: Selection Rules}
\begin{itemize}
    \item \marginnote{11/28:}Last in-class assessment on Friday. Same rules and format as last time.
    \item Review session this Wednesday; submit questions via the GForm. Submit Wuttig questions by the end of the day today. If there are no questions, Wuttig will treat the class as office hours.
    \item Last time: Electronic spectroscopy. We outlined how to rationalize, assign, and determine $B$ and $\Delta_o$ values for experimental electronic transitions for octahedral complexes.
    \item Today: We will derive the basis for the selection rules.
    \item \textbf{Selection rules}: All of the lines on the TS diagrams inform us of possible transitions. But are they probable?
    \item When deriving these, something we want to consider is oscillator strength. This is given by
    \begin{equation*}
        \int_0^\infty\varepsilon(\nu)\dd\nu \propto \mel{\Psi_\text{gs}}{M}{\Psi_\text{es}}^2
    \end{equation*}
    \begin{itemize}
        \item This is the integral under the spectrum of your compound.
        \item Conclusion: The oscillator strength is directly proportional to the transition moment integral.
    \end{itemize}
    \item The ground state and excited state have electronic, spin, and vibrational contributions.
    \item We now dissect $M$.
    \begin{itemize}
        \item By the B-O approximation, we can separate out the total wave function into the electronic contribution and the vibrational contribution because we assume the time-scale of the electronic and vibrational transitions are distinct.
        \begin{equation*}
            \Psi = \Psi_\text{e}\Psi_\text{v}
        \end{equation*}
        \begin{itemize}
            \item Rationale for separating electronic and vibrational motion: Experimental observation of vertical electronic transitions (see Figure \ref{fig:FC}) means that electrons move before the nuclear coordinate can significantly change.
        \end{itemize}
        \item Thus, we can dissect
        \begin{equation*}
            \hat{M} = \braket{\Psi_\text{esv}}{\Psi_\text{gsv}}\mel{\Psi_\text{ese}}{\hat{\mu}}{\Psi_\text{gse}}\braket{\Psi_\text{ess}}{\Psi_\text{gss}}
        \end{equation*}
        where we write $v$ for vibrational, $e$ for electronic, and $s$ for spin.
        \item $\hat{\mu}$ is the electronic polarization operator. It transforms as $xyz$, meaning that it has $u$ (ungerade) symmetry as a linear operator, similar to the $p$-orbitals.
    \end{itemize}
    \item We now dissect the components of $\hat{M}$ further.
    \item Spin.
    \begin{itemize}
        \item The spin contribution is nonzero iff $\Delta s=0$ (i.e., no change in spin). This justifies the \textbf{spin selection rule}.
    \end{itemize}
    \item \textbf{Spin selection rule}: We \emph{must} have the same spin state in the ground and excited states.
    \item Vibration.
    \begin{itemize}
        \item Typically, an electronic excitation will produce a vibrationally hot excited state (recall hot bands).
        \item Is the hot excited state the result of relaxation after electronic excitation, or will there be a hot excited state?
        \begin{itemize}
            \item We go to a hot vibrational excited state.
            \item For example, in going from $E=0$ to $E=1$, we will often go from $v=0$ to $v=2$ or something (see Figure \ref{fig:FC} again).
        \end{itemize}
        \item This can relax some selection rules and make transitions more probable.
    \end{itemize}
    \item Electronic.
    \begin{itemize}
        \item The component is nonzero iff the direct product contains $a_{1g}$ (i.e., is even over all space).
    \end{itemize}
    \item Electronic component in the context of $d$-$d$ transitions.
    \begin{itemize}
        \item For a $d$-$d$ transition, $\hat{\mu}$ is $u$ but the $d$ orbitals are $g$ (they are symmetric with respect to inversion). Taking the direct product yields $u$ symmetry.
        \begin{equation*}
            g_\text{es}\times u\times g_\text{gs} = u
        \end{equation*}
        \item It follows that the $d$-$d$ transitions are forbidden (this is the \textbf{Laporte selection rule}).
        \item Even though these transitions are formally Laporte forbidden, coupling with the hot bands in the vibrational state makes these become allowed.
        \item Wuttig will not go through all the math.
        \item Note: Coupling of $g_\text{es}$ with $u_\text{vib}$ can relax the rule!
    \end{itemize}
    \item \textbf{Laporte selection rule}: $d$-$d$ transitions are allowed iff the overall direct product has gerade symmetry. \emph{Also known as} \textbf{orbital selection rule}.
    \item Rank the following compounds in terms of oscillation strength.
    \begin{figure}[h!]
        \centering
        \footnotesize
        \begin{subfigure}[b]{0.3\linewidth}
            \centering
            \chemleft{[}
                \chemfig{Co(<:[:20]NH_3)(-[2]Cl)(<:[:160]H_3N)(<[:-160]H_3N)(-[6]Cl)(<[:-20]NH_3)}
            \chemright{]^+}
            \caption{\emph{trans} isomer.}
            \label{fig:oscStrengtha}
        \end{subfigure}
        \begin{subfigure}[b]{0.3\linewidth}
            \centering
            \chemleft{[}
                \chemfig{Co(<:[:20]Cl)(-[2]Cl)(<:[:160]H_3N)(<[:-160]H_3N)(-[6]NH_3)(<[:-20]NH_3)}
            \chemright{]^+}
            \caption{\emph{cis} isomer.}
            \label{fig:oscStrengthb}
        \end{subfigure}
        \caption{Oscillation strength comparison.}
        \label{fig:oscStrength}
    \end{figure}
    \begin{itemize}
        \item Comparing the compounds: Same $d$-electron count, same ligand composition. Thus perhaps it really is something to do with the ligand field?
        \item The \emph{trans} isomer is more symmetric than the \emph{cis} isomer ($D_{4h}$ vs. $C_{2v}$).
        \item If we have a more symmetric field, we'll have a lower oscillation strength, and vice versa for the less symmetric field. This is because the less symmetric field is more $u$-like, leading to greater vibrational coupling.
    \end{itemize}
    \item LMCT and MLCT.
    \item One way to relax the Laporte selection rule is to couple with vibrationally hot excited states. Another way is to have the transition occur from another part of the molecule.
    \item LMCT transitions.
    \begin{figure}[h!]
        \centering
        \begin{subfigure}[b]{0.4\linewidth}
            \centering
            \includegraphics[width=\linewidth]{../ExtFiles/LMCTa.png}
            \caption{Spectrum.}
            \label{fig:LMCTa}
        \end{subfigure}
        \begin{subfigure}[b]{0.28\linewidth}
            \centering
            \begin{tikzpicture}
                \draw [densely dashed]
                    (0,0) -- ++(0:1)
                    (0,0) -- ++(45:1)
                    (0,0) -- ++(90:1)
                    (0,0) -- ++(180:1)
                    (0,0) -- ++(-90:1)
                ;
    
                \filldraw [semithick,fill=white,rotate=45] (0,0)
                    to[out=0,in=0,out looseness=0.5,in looseness=1.5] ++(0,0.65)
                    to[out=180,in=180,out looseness=1.5,in looseness=0.5] cycle
                ;
                \filldraw [semithick,fill=white,rotate=-135] (0,0)
                    to[out=0,in=0,out looseness=0.5,in looseness=1.5] ++(0,0.65)
                    to[out=180,in=180,out looseness=1.5,in looseness=0.5] cycle
                ;
                \filldraw [semithick,fill=grz,rotate=-45] (0,0)
                    to[out=0,in=0,out looseness=0.5,in looseness=1.5] ++(0,0.65)
                    to[out=180,in=180,out looseness=1.5,in looseness=0.5] cycle
                ;
                \filldraw [semithick,fill=grz,rotate=135] (0,0)
                    to[out=0,in=0,out looseness=0.5,in looseness=1.5] ++(0,0.65)
                    to[out=180,in=180,out looseness=1.5,in looseness=0.5] cycle
                ;
    
                \draw [densely dashed] (0,0) -- ++(-135:1);
    
                \filldraw [semithick,fill=white,rotate around={90:(0,1)}] (0,1)
                    to[out=0,in=0,out looseness=0.5] ++(0,0.65)
                    to[out=180,in=180,in looseness=0.5] cycle
                ;
                \filldraw [semithick,fill=grz,rotate around={-90:(0,1)}] (0,1)
                    to[out=0,in=0,out looseness=0.5] ++(0,0.65)
                    to[out=180,in=180,in looseness=0.5] cycle
                ;
    
                \draw [semithick,-stealth] (-0.75,1) to[bend right=70] (-0.6,0.5);
            \end{tikzpicture}
            \caption{$\pi$-donation.}
            \label{fig:LMCTb}
        \end{subfigure}
        \begin{subfigure}[b]{0.3\linewidth}
            \centering
            \begin{tikzpicture}[
                da/.pic={\node{$\upharpoonleft\hspace{-1mm}\downharpoonright$};}
            ]
                \Large
                \draw [ultra thick]
                    (-0.55,2)    --         ++(0.5,0) ++(0.1,0) --         ++(0.5,0)
                    (-0.85,1.5)  -- pic{da} ++(0.5,0) ++(0.1,0) -- pic{da} ++(0.5,0) ++(0.1,0) -- pic{da} ++(0.5,0)
                    (-0.85,0.3)  -- pic{da} ++(0.5,0) ++(0.1,0) -- pic{da} ++(0.5,0) ++(0.1,0) -- pic{da} ++(0.5,0)
                    (-0.85,-0.3) -- pic{da} ++(0.5,0) ++(0.1,0) -- pic{da} ++(0.5,0) ++(0.1,0) -- pic{da} ++(0.5,0)
                    (-0.85,-1.2) -- pic{da} ++(0.5,0) ++(0.1,0) -- pic{da} ++(0.5,0) ++(0.1,0) -- pic{da} ++(0.5,0)
                    (-0.85,-1.8) -- pic{da} ++(0.5,0) ++(0.1,0) -- pic{da} ++(0.5,0) ++(0.1,0) -- pic{da} ++(0.5,0)
                ;
    
                \footnotesize
                \node [right] at (1,2)    {\ce{M-L\sigma^*}, $e_g$};
                \node [right] at (1,1.5)  {\ce{M-L\pi^*}, $t_{2g}$};
                \node [right] at (1,0)    {$t_{1u}(\pi)$ / $t_{2u}(\pi)$};
                \node [right] at (1,-1.5) {$t_{1u}(\sigma)$};

                \draw [-stealth] (-1,0) -- ++(0,2);
                \draw [-stealth] (-1.3,-1.5) -- ++(0,3.5);
    
                % \path (-3.1,0) -- (3.1,0);
            \end{tikzpicture}
            \caption{MO diagram.}
            \label{fig:LMCTc}
        \end{subfigure}
        \caption{LMCT dynamics.}
        \label{fig:LMCT}
    \end{figure}
    \begin{itemize}
        \item Figure \ref{fig:LMCTa} depicts the spectrum for \ce{[PtBr6]^2-}.
        \item All ligands are $\pi$-donating.
        \item If we think about the MO parentage (ligand $p$-orbital and metal $d$-orbital), we can have a charge transfer from the ligand to the metal because of the MO diagram (see Figure \ref{fig:LMCTb}; Wuttig appears to draw it as antibonding?? Reversed $p$-orbital sign??).
        \item All of the orbitals below the frontier $t_{2g}$ \ce{M-L\pi^*} set are completely occupied, and what we're observing is the transition from those lower-lying orbitals, up, as illustrated in Figure \ref{fig:LMCTc}.
        \begin{itemize}
            \item Notice that the transitions in this MO diagram exactly mirror those in the spectrum in Figure \ref{fig:LMCTa}.
            \item Possible inconsistency??: Figure \ref{fig:LMCTa} is consistent with Figure \ref{fig:LMCTc}, but according to Figure \ref{fig:MOsML6pi}, $t_{2u}(\pi)$ should be degenerate with $t_{1g}(\pi)$, not $t_{1u}(\pi)$. Overall actually, the orbitals don't look entirely consistent. What modifications are we making?
        \end{itemize}
        \item LMCT is Laporte allowed because the ligand $p$ orbital (ground state) is ungerade and the metal $d$ orbital (excited state) is gerade. Mathematically,
        \begin{equation*}
            g_\text{es}\times u\times u_\text{gs} = g
        \end{equation*}
        \item Molar extinction coefficients are \emph{very} high (tens of thousands vs. single-digit $d$-$d$ transitions).
    \end{itemize}
    \item Question: Rank the energy of the LMCT transitions among \ce{[OsBr6]^2-}, \ce{[OsCl6]^2-}, and \ce{[OsI6]^2-}.
    \begin{itemize}
        \item Largest electronegativity (chloride) leads to the largest LMCT transition; smallest leads to smallest (iodide).
        \item This is because larger electronegativity leads to lower bonding orbitals and thus a higher energy jump up to the frontier orbitals.
    \end{itemize}
    \item MLCT transitions.
    \begin{figure}[h!]
        \centering
        \begin{subfigure}[b]{0.4\linewidth}
            \centering
            \includegraphics[width=\linewidth]{../ExtFiles/MLCTa.png}
            \caption{Spectrum.}
            \label{fig:MLCTa}
        \end{subfigure}
        \begin{subfigure}[b]{0.28\linewidth}
            \centering
            \begin{tikzpicture}
                \draw [densely dashed]
                    (0,0) -- ++(0:1)
                    (0,0) -- ++(45:1)
                    (0,0) -- ++(90:1)
                    (0,0) -- ++(180:1)
                    (0,0) -- ++(-90:1)
                ;
    
                \filldraw [semithick,fill=white,rotate=45] (0,0)
                    to[out=0,in=0,out looseness=0.5,in looseness=1.5] ++(0,0.65)
                    to[out=180,in=180,out looseness=1.5,in looseness=0.5] cycle
                ;
                \filldraw [semithick,fill=white,rotate=-135] (0,0)
                    to[out=0,in=0,out looseness=0.5,in looseness=1.5] ++(0,0.65)
                    to[out=180,in=180,out looseness=1.5,in looseness=0.5] cycle
                ;
                \filldraw [semithick,fill=grz,rotate=-45] (0,0)
                    to[out=0,in=0,out looseness=0.5,in looseness=1.5] ++(0,0.65)
                    to[out=180,in=180,out looseness=1.5,in looseness=0.5] cycle
                ;
                \filldraw [semithick,fill=grz,rotate=135] (0,0)
                    to[out=0,in=0,out looseness=0.5,in looseness=1.5] ++(0,0.65)
                    to[out=180,in=180,out looseness=1.5,in looseness=0.5] cycle
                ;
    
                \draw [densely dashed] (0,0) -- ++(-135:1);
    
                \filldraw [semithick,fill=white,rotate around={90:(0,1)}] (0,1)
                    to[out=0,in=0,out looseness=0.5] ++(0,0.65)
                    to[out=180,in=180,in looseness=0.5] cycle
                ;
                \filldraw [semithick,fill=grz,rotate around={-90:(0,1)}] (0,1)
                    to[out=0,in=0,out looseness=0.5] ++(0,0.65)
                    to[out=180,in=180,in looseness=0.5] cycle
                ;
                \filldraw [semithick,fill=white,rotate around={-90:(0,1.4)}] (0,1.4)
                    to[out=0,in=0,out looseness=0.5] ++(0,0.65)
                    to[out=180,in=180,in looseness=0.5] cycle
                ;
                \filldraw [semithick,fill=grz,rotate around={90:(0,1.4)}] (0,1.4)
                    to[out=0,in=0,out looseness=0.5] ++(0,0.65)
                    to[out=180,in=180,in looseness=0.5] cycle
                ;
    
                \draw [semithick,stealth-] (0.75,1) to[bend left=70] (0.6,0.5);
            \end{tikzpicture}
            \caption{$\pi$-acceptance.}
            \label{fig:MLCTb}
        \end{subfigure}
        \begin{subfigure}[b]{0.3\linewidth}
            \centering
            \begin{tikzpicture}[
                da/.pic={\node{$\upharpoonleft\hspace{-1mm}\downharpoonright$};}
            ]
                \Large
                \draw [ultra thick]
                    (-0.85,1)    --         ++(0.5,0) ++(0.1,0) --         ++(0.5,0) ++(0.1,0) --         ++(0.5,0)
                    (-0.85,0.5)  --         ++(0.5,0) ++(0.1,0) --         ++(0.5,0) ++(0.1,0) --         ++(0.5,0)
                    (-0.55,-1)   --         ++(0.5,0) ++(0.1,0) --         ++(0.5,0)
                    (-0.85,-1.5) -- pic{da} ++(0.5,0) ++(0.1,0) -- pic{da} ++(0.5,0) ++(0.1,0) -- pic{da} ++(0.5,0)
                ;
    
                \footnotesize
                \node [right] at (1,1)  {$t_{1u}(\pi^*)$};
                \node [right] at (1,0.5)  {$t_{2u}(\pi^*)$};
                \node [right] at (1,-1)   {\ce{M-L\sigma^*}, $e_g$};
                \node [right] at (1,-1.5) {\ce{M-L\pi}, $t_{2g}$};

                \draw [-stealth] (-1,-1.5)   -- ++(0,2);
                \draw [-stealth] (-1.3,-1.5) -- ++(0,2.5);
    
                \path (-3.1,0) -- (3.1,0);
            \end{tikzpicture}
            \caption{MO diagram.}
            \label{fig:MLCTc}
        \end{subfigure}
        \caption{MLCT dynamics.}
        \label{fig:MLCT}
    \end{figure}
    \begin{itemize}
        \item Figure \ref{fig:MLCTa} depicts the spectrum for \ce{[Cr(CO)6]^3+}.
        \item All ligands are $\pi$-accepting
        \item Once again, we get very large extinction coefficients, so we must be relaxing the Laporte selection rule.
        \begin{itemize}
            \item Laporte allowed due to transitions occurring from gerade ground states to ungerade excited states.
        \end{itemize}
        \item We excite electrons from the metal to the ligand in the MO diagram, resulting in two transitions in both the diagram (Figure \ref{fig:MLCTc}) and the spectrum (Figure \ref{fig:MLCTa}).
    \end{itemize}
    \item Technological applications of MLCT.
    \begin{itemize}
        \item The transitions are so allowed that we get a lot of applications.
    \end{itemize}
    \item The Gr\"{a}tzel Cell (by Michael Gr\"{a}tzel at EPFL). You can't store energy, but you can take light and generate an electric current.
    \begin{figure}[h!]
        \centering
        \begin{subfigure}[b]{0.49\linewidth}
            \centering
            \includegraphics[width=0.9\linewidth]{../ExtFiles/GratzelCella.png}
            \caption{Overall design.}
            \label{fig:GratzelCella}
        \end{subfigure}
        \begin{subfigure}[b]{0.49\linewidth}
            \centering
            \includegraphics[width=0.9\linewidth]{../ExtFiles/GratzelCellb.png}
            \caption{MLCT step.}
            \label{fig:GratzelCellb}
        \end{subfigure}
        \caption{Gr\"{a}tzel Cell mechanism.}
        \label{fig:GratzelCell}
    \end{figure}
    \begin{itemize}
        \item A bit like artificial photosynthesis. We can't generate fuel like in a fuel cell, but we can generate current.
        \item Uses a \ce{Ru(SCN)3L} complex. Experimentally, this molecule has a very deep MLCT charge transfer band, allowing it to absorb into the red and IR spectrum (needed for direct sunlight to electricity conversion).
        \item The compound is anchored onto a \ce{TiO2} (semiconducting) surface. The band gap is about \SI{3}{\electronvolt}. The \textbf{conduction band} (LUMO for an extended solid) is poised at a level such that when you photoexcite \ce{Ru^{II}}, oxidizing via MLCT \ce{Ru^{II} -> Ru^{III}} and reducing the \ce{tpy} ligand, you get an electron in an orbital (on \ce{tpy-}) that is higher than the \ce{TiO2} conduction band.
        \item Thus, you can dump the electron into \ce{TiO2}.
        \item \ce{TiO2} is connected to a conducting glass (usually fluorine-doped tin oxide) from which you can harvest the electron and use it for electricity.
        \item An additional redox mediator is used in solution to harvest electrons at a cathode and regenerate \ce{Ru^2+}; this is essential in order to be able to do the same thing again (think salt bridge). The mediator goes from
        \begin{equation*}
            \ce{I3- -> 3I-}
        \end{equation*}
        \item The lifetimes aren't that great.
        \item Selling point is that it is made out of very cheap materials, and is efficient because it makes use of very highly allowed MLCTs.
    \end{itemize}
    \item Photoredox catalysis.
    \begin{figure}[h!]
        \centering
        \begin{subfigure}[b]{0.3\linewidth}
            \centering
            \includegraphics[width=0.7\linewidth]{../ExtFiles/photoredoxCatala.png}
            \caption{\ce{[Ru(bpy)3]^2+} light absorption.}
            \label{fig:photoredoxCatala}
        \end{subfigure}
        \begin{subfigure}[b]{0.68\linewidth}
            \centering
            \includegraphics[width=0.9\linewidth]{../ExtFiles/photoredoxCatalb.png}
            \caption{Discovery mechanism: Quenching with \ce{MV^2+}.}
            \label{fig:photoredoxCatalb}
        \end{subfigure}
        \caption{Photoredox catalysis.}
        \label{fig:photoredoxCatal}
    \end{figure}
    \begin{itemize}
        \item Organic chemists have usurped it for catalysis, but it all came from inorganic chemistry.
        \item Whitten and Meyer established that \ce{Ru(bpy)3^2+} complexes have charge transfer processes, that the excited electron can reduce various organic compounds, and that the hole left behind can be refilled.
        \begin{equation*}
            \underbrace{\ce{[Ru(bpy)3]^2+}}_{d\pi^6} \ce{->[$h\nu$]} \underbrace{\ce{[Ru(bpy^{\cdot -})(bpy)2]^{2+*}}}_{d\pi^5{\pi^*}^1}
        \end{equation*}
        \begin{itemize}
            \item Discovered this by quenching the reaction with methyl viologen (\ce{MV^2+}).
        \end{itemize}
        \item The seminal papers are the \textbf{Meyer papers}.
        \item We improve photoredox catalysis further by thinking about the relative energy levels between where the $d$ electrons are coming from and where they're going.
        \begin{itemize}
            \item To this end, people have put in tons of \ce{CF3} groups to make a more powerful reductant.
        \end{itemize}
    \end{itemize}
    \item \textbf{Meyer papers}: The two papers \textcite{bib:Meyer1974} and \textcite{bib:Meyer2013}.
\end{itemize}



\section{Second Exam Review}
\begin{itemize}
    \item \marginnote{11/30:}Wuttig interpreted everyone's GForm questions as, "we need to go over PSet 4."
    \item Problem 1 is a descent in symmetry problem.
    \item First thing to note: You have a \ce{Re-Re} center with octahedral geometry about each \ce{Re} atom.
    \item We want to build the MO diagram to convince ourselves that $\text{B.O.}=1$.
    \item Descend in symmetry from \ce{Re(CO)6} $O_h$ to square pyramidal \ce{Re(CO)5}, which is $C_{4v}$. What happens to the orbitals?
    \begin{itemize}
        \item \ce{CO} is a $\pi$-acceptor.
        \item Thus, the $e_g$ set has \ce{M-L\sigma^*} parentage, and the $t_{2g}$ set has \ce{M-L\pi} parentage.
        \item Labeling the orbitals allows you to investigate them individually. In particular, $d_{z^2}$ decreases in energy and $d_{x^2-y^2}$ stays the same; get their Mulliken symbols from the $C_{4v}$ character table. Less $\pi$-backbonding gives you a raised $e$ set ($d_{xz,yz}$) and a conserved $b_2$ $d_{xy}$.
    \end{itemize}
    \item We assume that we are engaging this fragment with itself.
    \begin{itemize}
        \item Thus, we write the MO hierarchy on both sides of our MO diagram, and then we mix these orbitals similarly to Figure \ref{fig:MOsRe2Cl8}.
        \item Based on the parentage, we can now determine what's $\sigma$, $\pi$, and $\delta$.
        \item We now fill in 14 electrons, corresponding to two neutral $d^7$ \ce{Re} atoms.
    \end{itemize}
    \item Descent in symmetry: Start from a molecule you understand, and then do systematic distortions on it.
    \item Outside of some Mulliken symbol errors, it looks like I got the question right!
    \item Can we scale this differently like I did?
    \begin{itemize}
        \item We can scale this any way we want, as long as we get a bond order of 1 and a splitting ordering of $\sigma>\pi>\delta$.
    \end{itemize}
    \item So antibonding orbitals can be lower in energy than bonding orbitals?
    \begin{itemize}
        \item Yes.
    \end{itemize}
    \item Problem 2 asks us to consider a $d^4$ $O_h$ compound and a $d^6$ $T_d$ diagram. In sum, it asks us to draw a correlation diagram and walks you through how to do that.
    \item 1. Determine the ground state free ion term.
    \item We can do so by maximizing spin $M_S=\sum m_s$ and $M_L=\sum m_\ell$.
    \item Label a free ion set of 5 $d$-orbitals with their $m_\ell$ values and fill in electrons.
    \begin{itemize}
        \item This gives $L=2+1+0-1=2$ and $S=\frac{1}{2}+\frac{1}{2}+\frac{1}{2}+\frac{1}{2}=2$, yielding ${}^5D$ as our free ion term.
        \item Doing this for the other case actually also gets us the same answer.
    \end{itemize}
    \item 2. What is the splitting of the ground state free ion term in a weak ligand field?
    \begin{itemize}
        \item We want to determine how ${}^5D$ splits in a weak (octahedral) field. Take $L=2$ and know that the representation can be approximated by the quadratic functions ($P$ is linear functions, $S$ is totally symmetric, $F$ is cubic). Thus, from the $O_h$ character table, we have $T_{2g}$ and $E_g$. The spin multiplicity of 5 carries over.
        \item Very similar to the above, but from the $T_d$ character table, we get $T_2$ and $E$.
    \end{itemize}
    \item We don't know the placement/relative energies of the split terms; we just want to know that it splits.
    \begin{itemize}
        \item In terms of energy ranking, we just need to be able to find the ground state of the free ion term (i.e., by knowing that it is the lowest in energy according to Hund's rules).
    \end{itemize}
    \item 3. The ground state strong-field electron configuration.
    \item Strong field means we want to minimize spin multiplicity across the two levels. Our electron configuration for $d^4$ is is ${t_{2g}}^4$. For $d^6$ $T_d$, we have $e^4{t_2}^2$.
    \begin{itemize}
        \item Wuttig says we want to minimize spin multiplicity??
    \end{itemize}
    \item 4. Split the ground state strong-field electron configuration into strong-field terms.
    \item We can take the direct product of the electrons in the unpaired set because the interaction between them will split the energetics of the strong field term.
    \item Take the direct product of $t_{2g}\times t_{2g}$ and know that the characters are
    \begin{table}[h!]
        \centering
        \small
        \renewcommand{\arraystretch}{1.2}
        \begin{tabular}{c|cccccccccc}
            $O_h$ & $E$ & $8C_3$ & $6C_2$ & $6C_4$ & $3C_2$ & $i$ & $6S_4$ & $8S_6$ & $3\sigma_h$ & $6\sigma_d$\\
            \hline
            $T_{2g}$ & 3 & 0 & 1 & $-1$ & $-1$ & 3 & $-1$ & 0 & $-1$ & 1\\
            $t_{2g}\times t_{2g}$ & 9 & 0 & 1 & 1 & 1 & 9 & 1 & 0 & 1 & 1
        \end{tabular}
        \caption{Representation for the unpaired electrons of an $O_h$ $d^4$ complex.}
        \label{tab:Ohd4RR}
    \end{table}
    \item Reduce to
    \begin{equation*}
        \Gamma = A_{1g}+E_g+T_{1g}+T_{2g}
    \end{equation*}
    \begin{itemize}
        \item Thus, the strong field ${t_{2g}}^4$ can be split into the above states.
    \end{itemize}
    \item We don't need to put spin multiplicities on the Mulliken symbols because doing so requires additional descent in symmetry analysis beyond we did not do in class.
    \item ${t_{2g}}^4$ splits into four terms; are these excited states?
    \begin{itemize}
        \item These are not excited states per se, it's just that there is finer structure within the strong field electron configuration.
    \end{itemize}
    \item With these four splittings, we don't necessarily need to fill the electrons into the splitting levels? Do different ones of these levels have varying occupations?
    \begin{itemize}
        \item Think about these as being electronic energy levels that are useful for TS diagrams, but not for MO diagrams.
    \end{itemize}
    \item Are the terms we came up with in Problem 2(3) the ground state strong field terms?
    \begin{itemize}
        \item Yes.
        \item This is still related to ${}^4F$ and the associated weak field things, but we do this correlation at the very end.
        \item We do the correlation using the two axioms (1-to-1 correlation and same spin modes [??]).
    \end{itemize}
    \item If we have an excited state $t_{2g}\to e_g$ transition, could we take the direct product $t_{2g}\times e_g$?
    \begin{itemize}
        \item Yes.
        \item Not intuitive, though, because there are more possibilities.
    \end{itemize}
    \item Why do we take a two-fold direct product instead of a four-fold direct product for ${t_{2g}}^4$?
    \begin{itemize}
        \item We must consider the physical origin of this mathematical calculation; we are determining the ways those two \emph{unpaired} electrons can couple.
        \item To reiterate, even though we have more electrons (four), we take the direct product twice to describe the two electrons that are standing alone in the strong field splitting.
        \item We assume that all paired electrons cancel.
    \end{itemize}
    \item Where did we find for the $T_d$ geometry that the $e$ orbitals are lower in energy than the $t_2$ orbitals (inverse of $O_h$ case)?
    \begin{itemize}
        \item That comes from crystal field splitting; four ligands on the corners of the tetrahedra shows that the $e$ orbitals are more effective at overlapping than $t_2$, and hence are more raised.
    \end{itemize}
    \item Same schenanigans on $t_2\times t_2$. We get
    \begin{table}[H]
        \centering
        \small
        \renewcommand{\arraystretch}{1.2}
        \begin{tabular}{c|ccccc}
            $O_h$ & $E$ & $8C_3$ & $3C_2$ & $6S_4$ & $6\sigma_d$\\
            \hline
            $T_2$ & 3 & 0 & $-1$ & $-1$ & 1\\
            $t_2\times t_2$ & 9 & 0 & 1 & 1 & 1\\
        \end{tabular}
        \caption{Representation for the unpaired electrons of a $T_d$ $d^6$ complex.}
        \label{tab:Tdd6RR}
    \end{table}
    \item Reduce to
    \begin{equation*}
        \Gamma = A_1+E+T_1+T_2
    \end{equation*}
    \item For the reduction, why do we multiply $t_2\times t_2$?
    \begin{itemize}
        \item Remember the electron hole formalism. $6-4$ electrons gives 2 holes.
        \item We will not have to do more that 2 electrons ever in this class. It's beyond the scope, and we can leave that to the physicists. 2 electrons is possible, though.
    \end{itemize}
    \item 5. Determine if any weak ligand field term to strong-field term exhibit correlation for terms derived for the ground state.
    \item Draw out the $x$-axis for the correlation diagram. Left to right, it reads free ions, weak field, strong field terms, and strong field.
    \item Make sure to indicate that your solution works analogously for both the $d^4$ in $O_h$ case and the $d^6$ in $T_d$ case. The only difference between the $O_h$ and $T_d$ labels is the presence or absence of $g$, so we can indicate this in our single correlation diagram by putting parentheses around each subscript $g$.
    \item But if we wanted to correlate anything, we would need a spin multiplicity of 5 on the strong field terms.
    \item 6. Indicate why we can use the same energy correlation diagram to describe $d^4$ in $O_h$ and $d^6$ in $T_d$ ligand field.
    \item Comes directly from the electron hole formalism. $T_d$ inverts the energies of the $E,T_2$ sets; changes the number of electrons and positrons while keeping the environment mostly the same.
    \item Now to Question 3.
    \item 1. Ground state term symbol.
    \item ${}^4F$ is derived much the same way as in problem 2.
    \item Similar to the above, we get a certain configuration. from which we derive a term symbol, specifically ${}^4F$ for a $d^7$ complex. This makes sense since we have a ${}^4F$ label at the bottom-left corner.
    \item 2. What are the predicted spectral bands based on selection rules?
    \item We can get
    \begin{align*}
        {}^4F({}^4T_1) &\to {}^4T_2\\
        &\to {}^4A_2\\
        &\to {}^4T_1({}^4P)
    \end{align*}
    \item We got this by reading the lines off of the TS diagram and looking for ones with the same spin (spin selection rule \emph{must} be satisfied, unlike the Laporte selection rule).
    \begin{itemize}
        \item This is purely theoretical; there are no experiments.
        \item You don't know if you're high-spin or low-spin; you don't know anything. But you should start your transitions with the free ion term (${}^4F$ in this case), regardless!
    \end{itemize}
    \item 3. Calculate the $\Delta_o$'s.
    \item You need to treat the Tanabe-Sugano diagram as a working curve. If it's a working curve, then you can take a ratio of the two energy transitions, namely
    \begin{equation*}
        \frac{v_2}{v_1} = \frac{22,000}{11,300}
        = 1.95
    \end{equation*}
    \item We do indeed assume that the top two transitions listed above are $v_1,v_2$.
    \item You need to do a sliding scale and figure out where you slide to get the right ratio.
    \item Note: We're wildly guessing here; there will be huge errors in the numbers we get. Using a system of equations, if applicable, is slightly more elegant.
    \item Take $E/B=17$ for the lower energy. Then we have $E/B=28$ for the higher energy by following vertical and horizontal lines on the graph. Since we know the energies, we can calculate the Racah parameter at this point. The two values for the Racah parameter should be the same; if they're not, take the average.
    \item It turns out that the bipy ligand is not as strong as we think it is.
    \item Will there be a definition matching question again?
    \begin{itemize}
        \item There will be some gimme points and some thinking points.
    \end{itemize}
\end{itemize}




\end{document}