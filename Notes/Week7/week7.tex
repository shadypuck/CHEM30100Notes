\documentclass[../notes.tex]{subfiles}

\pagestyle{main}
\renewcommand{\chaptermark}[1]{\markboth{\chaptername\ \thechapter\ (#1)}{}}
\setcounter{chapter}{6}

\begin{document}




\chapter{Bonding in Coordination Complexes}
\section[Coordination Complexes: \texorpdfstring{$\pi$}{TEXT} Bonding and all-\texorpdfstring{$\sigma$}{TEXT} Distortions]{Coordination Complexes: \texorpdfstring{$\bm{\pi}$}{TEXT} Bonding and all-\texorpdfstring{$\bm{\sigma}$}{TEXT} Distortions}
\begin{itemize}
    \item \marginnote{11/7:}Video lecture Wed and Fri; watch by Mon and have questions.
    \item Last time: MO diagram for an \ce{ML6} coordination complex whose ligands only engage in $\sigma$ interactions.
    \item Today: The case where the ligands also have $\pi$ orbitals.
    \item As per usual, we follow the procedure from Lecture 6.1.
    \begin{enumerate}
        \item Point group: $O_h$.
        \item Assigning coordinate axes is a bit more complicated here, but the following will work.
        \begin{figure}[h!]
            \centering
            \includegraphics[width=0.2\linewidth]{../ExtFiles/piOrientationsML6.png}
            \caption{\ce{ML6} $\pi$ $xyz$ coordinates.}
            \label{fig:xyzML6pi}
        \end{figure}
        \begin{itemize}
            \item We have two orthogonal $p\pi$ bonds.
            \item The arrows indicate the directional phase of the $p$ orbitals.
        \end{itemize}
        \item Let's create a representation for the 12 $p$ orbitals capable of $\pi$ bonding.
        \begin{table}[h!]
            \centering
            \small
            \renewcommand{\arraystretch}{1.2}
            \begin{tabular}{c|cccccccccc}
                $O_h$ & $E$ & $8C_3$ & $6C_2$ & $6C_4$ & $3C_2$ & $i$ & $6S_4$ & $8S_6$ & $3\sigma_h$ & $6\sigma_d$\\
                \hline
                $\Gamma_{12\pi}$ & 12 & 0 & 0 & 0 & $-4$ & 0 & 0 & 0 & 0 & 0\\
            \end{tabular}
            \caption{Representation for the $p\pi$ ligand orbitals of \ce{ML6}.}
            \label{tab:ML6pRR}
        \end{table}
        \item Reducing, we get
        \begin{equation*}
            \Gamma_{12\pi} = T_{1g}+T_{1u}+T_{2g}+T_{2u}
        \end{equation*}
        \item From Table \ref{tab:charTableOh}, the atomic orbitals of the metal transform as
        \begin{align*}
            s &\sim a_{1g}&
            p_x,p_y,p_z &\sim t_{1u}&
            d_{z^2},d_{x^2-y^2} &\sim e_g&
            d_{xz},d_{yz},d_{xy} &\sim t_{2g}
        \end{align*}
        \begin{itemize}
            \item Thus, the metal $p$ and $d_{xz},d_{yz},d_{xy}$ orbitals interact with the ligand $\pi$ SALCS.
        \end{itemize}
        \pagebreak
        \item Once again, we can "guess" the SALCs based on experience.
        \begin{figure}[H]
            \centering
            \begin{subfigure}[b]{0.24\linewidth}
                \centering
                \begin{tikzpicture}
                    \footnotesize
                    \draw
                        (0,0) coordinate (M) -- ++(90:4em)   coordinate (L1)
                        (0,0)                -- ++(-160:4em) coordinate (L2)
                        (0,0)                -- ++(-20:4em)  coordinate (L3)
                        (0,0)                -- ++(20:4em)   coordinate (L4)
                        (0,0)                -- ++(160:4em)  coordinate (L5)
                        (0,0)                -- ++(-90:4em)  coordinate (L6)
                    ;
        
                    \filldraw [semithick,fill=white] (M)
                        to[out=0,in=0,out looseness=0.5] ++(0,0.5)
                        to[out=180,in=180,in looseness=0.5] cycle
                    ;
                    \filldraw [semithick,fill=grz] (M)
                        to[out=0,in=0,out looseness=0.5] ++(0,-0.5)
                        to[out=180,in=180,in looseness=0.5] cycle
                    ;
                    \foreach \n in {L2,L3,L4,L5} {
                        \filldraw [semithick,fill=white] (\n)
                            to[out=0,in=0,out looseness=0.5] ++(0,0.3)
                            to[out=180,in=180,in looseness=0.5] cycle
                        ;
                        \filldraw [semithick,fill=grz] (\n)
                            to[out=0,in=0,out looseness=0.5] ++(0,-0.3)
                            to[out=180,in=180,in looseness=0.5] cycle
                        ;
                    }
                \end{tikzpicture}
                \caption{$t_{1u}$ bonding.}
                \label{fig:SALCML6pia}
            \end{subfigure}
            \begin{subfigure}[b]{0.24\linewidth}
                \centering
                \begin{tikzpicture}
                    \footnotesize
                    \draw
                        (0,0) coordinate (M) -- ++(90:4em)   coordinate (L1)
                        (0,0)                -- ++(-160:4em) coordinate (L2)
                        (0,0)                -- ++(-20:4em)  coordinate (L3)
                        (0,0)                -- ++(20:4em)   coordinate (L4)
                        (0,0)                -- ++(160:4em)  coordinate (L5)
                        (0,0)                -- ++(-90:4em)  coordinate (L6)
                    ;
        
                    \filldraw [semithick,fill=white,rotate=0,yscale=0.6] (M)
                        to[out=0,in=0,out looseness=0.5,in looseness=1.5] ++(0,0.65)
                        to[out=180,in=180,out looseness=1.5,in looseness=0.5] cycle
                    ;
                    \filldraw [semithick,fill=grz,rotate=-90] (M)
                        to[out=0,in=0,out looseness=0.5,in looseness=1.5] ++(0,0.65)
                        to[out=180,in=180,out looseness=1.5,in looseness=0.5] cycle
                    ;
                    \filldraw [semithick,fill=grz,rotate=90] (M)
                        to[out=0,in=0,out looseness=0.5,in looseness=1.5] ++(0,0.65)
                        to[out=180,in=180,out looseness=1.5,in looseness=0.5] cycle
                    ;
                    \filldraw [semithick,fill=white,rotate=180,yscale=0.6] (M)
                        to[out=0,in=0,out looseness=0.5,in looseness=1.5] ++(0,0.65)
                        to[out=180,in=180,out looseness=1.5,in looseness=0.5] cycle
                    ;
        
                    \foreach \n/\a in {L2/-110,L3/110,L4/70,L5/-70} {
                        \filldraw [semithick,fill=white,rotate=\a] (\n)
                            to[out=0,in=0,out looseness=0.5] ++(0,0.3)
                            to[out=180,in=180,in looseness=0.5] cycle
                        ;
                        \filldraw [semithick,fill=grz,rotate=\a] (\n)
                            to[out=0,in=0,out looseness=0.5] ++(0,-0.3)
                            to[out=180,in=180,in looseness=0.5] cycle
                        ;
                    }
                \end{tikzpicture}
                \caption{$t_{2g}$ bonding.}
                \label{fig:SALCML6pib}
            \end{subfigure}
            \begin{subfigure}[b]{0.24\linewidth}
                \centering
                \begin{tikzpicture}
                    \footnotesize
                    \draw
                        (0,0) coordinate (M) -- ++(90:4em)   coordinate (L1)
                        (0,0)                -- ++(-160:4em) coordinate (L2)
                        (0,0)                -- ++(-20:4em)  coordinate (L3)
                        (0,0)                -- ++(20:4em)   coordinate (L4)
                        (0,0)                -- ++(160:4em)  coordinate (L5)
                        (0,0)                -- ++(-90:4em)  coordinate (L6)
                    ;
        
                    \foreach \n/\a in {L2/-110,L3/-70,L4/70,L5/110} {
                        \filldraw [semithick,fill=white,rotate=\a] (\n)
                            to[out=0,in=0,out looseness=0.5] ++(0,0.3)
                            to[out=180,in=180,in looseness=0.5] cycle
                        ;
                        \filldraw [semithick,fill=grz,rotate=\a] (\n)
                            to[out=0,in=0,out looseness=0.5] ++(0,-0.3)
                            to[out=180,in=180,in looseness=0.5] cycle
                        ;
                    }
                \end{tikzpicture}
                \caption{$t_{1g}$ nonbonding.}
                \label{fig:SALCML6pic}
            \end{subfigure}
            \begin{subfigure}[b]{0.24\linewidth}
                \centering
                \begin{tikzpicture}
                    \footnotesize
                    \draw
                        (0,0) coordinate (M) -- ++(90:4em)   coordinate (L1)
                        (0,0)                -- ++(-160:4em) coordinate (L2)
                        (0,0)                -- ++(-20:4em)  coordinate (L3)
                        (0,0)                -- ++(20:4em)   coordinate (L4)
                        (0,0)                -- ++(160:4em)  coordinate (L5)
                        (0,0)                -- ++(-90:4em)  coordinate (L6)
                    ;
        
                    \foreach \n/\a in {L2/0,L3/180,L4/0,L5/180} {
                        \filldraw [semithick,fill=white,rotate=\a] (\n)
                            to[out=0,in=0,out looseness=0.5] ++(0,0.3)
                            to[out=180,in=180,in looseness=0.5] cycle
                        ;
                        \filldraw [semithick,fill=grz,rotate=\a] (\n)
                            to[out=0,in=0,out looseness=0.5] ++(0,-0.3)
                            to[out=180,in=180,in looseness=0.5] cycle
                        ;
                    }
                \end{tikzpicture}
                \caption{$t_{2u}$ nonbonding.}
                \label{fig:SALCML6pid}
            \end{subfigure}
            \caption{\ce{ML6} $\pi$ SALCs.}
            \label{fig:SALCML6pi}
        \end{figure}
        \begin{itemize}
            \item In addition to these, there are two more (oriented along the other orthogonal coordinate axes) for each IRR.
        \end{itemize}
        \item We can now make sense of the MO diagram.
        \begin{figure}[h!]
            \centering
            \includegraphics[width=0.6\linewidth]{../ExtFiles/MOsML6pi.png}
            \caption{\ce{ML6} $\pi$ MO diagram.}
            \label{fig:MOsML6pi}
        \end{figure}
        \begin{itemize}
            \item Wuttig compares this MO diagram to the all-$\sigma$ case (Figure \ref{fig:MOsML6}).
            \begin{itemize}
                \item The \ce{M}-\ce{L\sigma} and \ce{M}-\ce{L\sigma^*} distortions stay the same; we now just have \emph{additional} \ce{M}-\ce{L\pi} and \ce{M}-\ce{L\pi^*} distortions to consider.
                \item $\pi$-donating ligands, such as the fluoride ligands causing the interactions in Figure \ref{fig:MOsML6pi}, \emph{raise} the $t_{2g}$ set in energy; they contribute \emph{antibonding} interactions.
                \item Note that the \ce{L\pi} orbitals sit above the \ce{L\sigma} orbitals.
            \end{itemize}
            \item Wuttig probably expects us to be familiar with the \ce{M}-\ce{L\pi^*} notation.
        \end{itemize}
    \end{enumerate}
    \item It seems that now we're done. But wait: We have made an assumption that is not necessarily justified in every case.
    \begin{itemize}
        \item We have treated the ligand as a point particle with atomic orbitals that mirror the metal center (i.e., your typical $s$, $p$, etc. orbitals).
        \item This is justified in the case of fluoride (as in Figure \ref{fig:MOsML6pi}). But what about a ligand such as carbon monoxide? \ce{CO} certainly has molecular orbitals more complicated than the atomic orbitals of either carbon or oxygen alone, so is it still valid to treat it as a point particle with "atomic" orbitals?
        \item In fact, it is not, and we will now see how to treat that case.
    \end{itemize}
    \item As with fluoride, the frontier orbitals of \ce{CO} will be the ones that interact with the metal center.
    \item To draw SALCs of the interactions of these MOs with the metal center, we need to know what \emph{they} look like first. Fortunately, we have encountered the SALCs for heteronuclear diatomics before, and we can simply use these as our basis set to draw the overall SALCs.
    \begin{figure}[h!]
        \centering
        \begin{subfigure}[b]{0.24\linewidth}
            \centering
            \begin{tikzpicture}
                \footnotesize
                \draw
                    (0,0) coordinate (M) -- ++(90:4em)   coordinate (L1)
                    (0,0)                -- ++(-160:4em) coordinate (L2)
                    (0,0)                -- ++(-20:4em)  coordinate (L3)
                    (0,0)                -- ++(20:4em)   coordinate (L4)
                    (0,0)                -- ++(160:4em)  coordinate (L5)
                    (0,0)                -- ++(-90:4em)  coordinate (L6)
                ;
    
                \filldraw [semithick,fill=white] (M)
                    to[out=0,in=0,out looseness=0.5] ++(0,0.5)
                    to[out=180,in=180,in looseness=0.5] cycle
                ;
                \filldraw [semithick,fill=grz] (M)
                    to[out=0,in=0,out looseness=0.5] ++(0,-0.5)
                    to[out=180,in=180,in looseness=0.5] cycle
                ;
                \foreach \n in {L2,L3,L4,L5} {
                    \filldraw [semithick,fill=white] ($(M)!0.8!(\n)$)
                        to[out=0,in=0,out looseness=0.5] ++(0,0.3)
                        to[out=180,in=180,in looseness=0.5] cycle
                    ;
                    \filldraw [semithick,fill=grz] ($(M)!0.8!(\n)$)
                        to[out=0,in=0,out looseness=0.5] ++(0,-0.3)
                        to[out=180,in=180,in looseness=0.5] cycle
                    ;
                    \filldraw [semithick,fill=white] (\n)
                        to[out=0,in=0,out looseness=0.5] ++(0,-0.3)
                        to[out=180,in=180,in looseness=0.5] cycle
                    ;
                    \filldraw [semithick,fill=grz] (\n)
                        to[out=0,in=0,out looseness=0.5] ++(0,0.3)
                        to[out=180,in=180,in looseness=0.5] cycle
                    ;
                }
            \end{tikzpicture}
            \caption{$t_{1u}$ bonding.}
            \label{fig:SALCMCO6a}
        \end{subfigure}
        \begin{subfigure}[b]{0.24\linewidth}
            \centering
            \begin{tikzpicture}
                \footnotesize
                \draw
                    (0,0) coordinate (M) -- ++(90:4em)   coordinate (L1)
                    (0,0)                -- ++(-160:4em) coordinate (L2)
                    (0,0)                -- ++(-20:4em)  coordinate (L3)
                    (0,0)                -- ++(20:4em)   coordinate (L4)
                    (0,0)                -- ++(160:4em)  coordinate (L5)
                    (0,0)                -- ++(-90:4em)  coordinate (L6)
                ;
    
                \filldraw [semithick,fill=white,rotate=0,yscale=0.6] (M)
                    to[out=0,in=0,out looseness=0.5,in looseness=1.5] ++(0,0.65)
                    to[out=180,in=180,out looseness=1.5,in looseness=0.5] cycle
                ;
                \filldraw [semithick,fill=grz,rotate=-90] (M)
                    to[out=0,in=0,out looseness=0.5,in looseness=1.5] ++(0,0.65)
                    to[out=180,in=180,out looseness=1.5,in looseness=0.5] cycle
                ;
                \filldraw [semithick,fill=grz,rotate=90] (M)
                    to[out=0,in=0,out looseness=0.5,in looseness=1.5] ++(0,0.65)
                    to[out=180,in=180,out looseness=1.5,in looseness=0.5] cycle
                ;
                \filldraw [semithick,fill=white,rotate=180,yscale=0.6] (M)
                    to[out=0,in=0,out looseness=0.5,in looseness=1.5] ++(0,0.65)
                    to[out=180,in=180,out looseness=1.5,in looseness=0.5] cycle
                ;
    
                \foreach \n/\a in {L2/-110,L3/110,L4/70,L5/-70} {
                    \filldraw [semithick,fill=white,rotate=\a] ($(M)!0.8!(\n)$)
                        to[out=0,in=0,out looseness=0.5] ++(0,0.3)
                        to[out=180,in=180,in looseness=0.5] cycle
                    ;
                    \filldraw [semithick,fill=grz,rotate=\a] ($(M)!0.8!(\n)$)
                        to[out=0,in=0,out looseness=0.5] ++(0,-0.3)
                        to[out=180,in=180,in looseness=0.5] cycle
                    ;
                    \filldraw [semithick,fill=white,rotate=\a] (\n)
                        to[out=0,in=0,out looseness=0.5] ++(0,-0.3)
                        to[out=180,in=180,in looseness=0.5] cycle
                    ;
                    \filldraw [semithick,fill=grz,rotate=\a] (\n)
                        to[out=0,in=0,out looseness=0.5] ++(0,0.3)
                        to[out=180,in=180,in looseness=0.5] cycle
                    ;
                }
            \end{tikzpicture}
            \caption{$t_{2g}$ bonding.}
            \label{fig:SALCMCO6b}
        \end{subfigure}
        \caption{\ce{M(CO)6} $\pi$-bonding SALCs.}
        \label{fig:SALCMCO6}
    \end{figure}
    \begin{itemize}
        \item Wuttig draws the Figure \ref{fig:SALCMCO6a} interactions in-plane, too, though??
        \item Wuttig also doesn't draw the nonbonding ones, but they still exist??
        \item To reiterate, the $\pi^*$ orbitals of \ce{CO} will participate in $t_{1u}$ and $t_{2g}$ bonding interactions, and $t_{1g}$ and $t_{2u}$ nonbonding interactions just like the $p$ orbitals of \ce{F}; it is \emph{strictly} and \emph{solely} the basis set that we're changing.
    \end{itemize}
    \item An additional complication arises from the fact that the frontier orbitals of \ce{CO} are fundamentally different than those of \ce{F}.
    \begin{itemize}
        \item In particular, the frontier orbitals of \ce{F} are filled $\pi$-donating atomic orbitals, while \ce{CO} has a filled $\sigma$-donating frontier orbital (HOMO) and an unfilled $\pi^*$-accepting frontier orbital (LUMO).
        \item Thus, we need to consider the new \emph{energetics} of these orbitals as well.
        \item For a $\pi$-donating ligand such as monoatomic fluoride, the $\sigma$ and $\pi$ orbitals are degenerate.
        \begin{itemize}
            \item This is because all 18 ligand orbitals come from the $2p$ \emph{atomic} orbitals of fluorine.
        \end{itemize}
        \item However\dots
    \end{itemize}
    \item Are the $\sigma$ and $\pi$ orbitals of the $\pi$-accepting ligands degenerate in energy?
    \begin{itemize}
        \item They are not.
        \item This is because we are considering the interactions of nondegenerate ligand \emph{molecular} orbitals with the metal center.
        \item Evidence: We can inspect the photoelectron spectrum of our ligand (e.g., for \ce{CO}, we observe distinct peaks corresponding to its $\sigma$-donating and $\pi^*$-accepting orbitals).
        \item Note that ligands such as \ce{CO} still have filled $\pi$ MOs; it's just that these lie so low in energy that they don't interact with the metal center.
    \end{itemize}
    \item The consequence of this is that the $\pi$ ligand orbitals of a $\pi$-accepting ligand lie significantly higher in energy.
    \begin{itemize}
        \item In fact, they lie higher in energy than a metal's $d$ orbitals, meaning that the metal $t_{2g}$ set is now \ce{M}-\ce{L\pi} \emph{bonding} instead of \emph{antibonding} and hence lower in energy, leading to a greater $d$ orbital splitting.
    \end{itemize}
    \item All of this can be summarized by the MO diagram for a \ce{ML6} complex with $\pi$-accepting ligands.
    \begin{figure}[h!]
        \centering
        \includegraphics[width=0.6\linewidth]{../ExtFiles/MOsML6piAcc.png}
        \caption{\ce{ML6} $\pi$-accepting MO diagram.}
        \label{fig:MOsML6piAcc}
    \end{figure}
    \item Distortions from $\sigma$ interactions.
    \item Consider a \ce{ML6} complex with $\sigma$-only interactions, as discussed last class.
    \begin{itemize}
        \item Goal: Predict Jahn-Teller distortions from first principles.
        \item Two possible distortions: A tetragonal compression or a Jahn-Teller elongation.
        \begin{itemize}
            \item See Figure VI.10 of \textcite{bib:CHEM20100Notes}.
            \item Are both of these distortions not "Jahn-Teller" effects??
        \end{itemize}
        \item Either distortion changes the point group from $O_h$ to $D_{4h}$.
    \end{itemize}
    \item We now seek to build an MO diagram for the $t_{2g}$ and $e_g$ set of each distortion.
    \begin{table}[h!]
        \centering
        \small
        \renewcommand{\arraystretch}{1.2}
        \begin{tabular}{c|c|c}
             & $O_h$ & $D_{4h}$\\
            \hline
            $d_{x^2-y^2}$ & $e_g$ & $b_{1g}$\\
            $d_{z^2}$ & $e_g$ & $a_{1g}$\\
            $d_{xy}$ & $t_{2g}$ & $b_{2g}$\\
            $d_{yz}$ & $t_{2g}$ & $e_g$\\
            $d_{xz}$ & $t_{2g}$ & $e_g$\\
        \end{tabular}
        \caption{$d$-orbital symmetries in $O_h$ vs. $D_{4h}$.}
        \label{tab:dOrbOhD4h}
    \end{table}
    \begin{itemize}
        \item To begin, we determine the symmetries are the $d$ orbitals in the two point groups (see Table \ref{tab:dOrbOhD4h}).
        \item Since the $t_{2g}$ set is nonbonding in a $\sigma$-only complex, their energy doesn't change much, so the distorted $t_{2g}$ set is basically still degenerate. However, we do have to draw the $b_{2g}$ MO slightly higher (or lower, but we choose higher) since it has a different symmetry.
        \item The $e_g$ orbitals definitively split, going in different directions as the axial ligands compress or expand. In particular, compressing the axial ligands drives $d_{z^2}$ up (and $d_{x^2-y^2}$ down), and vice versa for expanding the axial ligands.
        \item Note that the original $e_g$ set is \ce{M}-\ce{L\sigma^*}, which is why compression drives up energy (more mixing between regions of opposite phases is not energetically favorable).
    \end{itemize}
\end{itemize}




\end{document}