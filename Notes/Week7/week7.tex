\documentclass[../notes.tex]{subfiles}

\pagestyle{main}
\renewcommand{\chaptermark}[1]{\markboth{\chaptername\ \thechapter\ (#1)}{}}
\setcounter{chapter}{6}

\begin{document}




\chapter{Bonding in Coordination Complexes}
\section[Coordination Complexes: \texorpdfstring{$\pi$}{TEXT} Bonding and all-\texorpdfstring{$\sigma$}{TEXT} Distortions]{Coordination Complexes: \texorpdfstring{$\bm{\pi}$}{TEXT} Bonding and all-\texorpdfstring{$\bm{\sigma}$}{TEXT} Distortions}
\begin{itemize}
    \item \marginnote{11/7:}Video lecture Wed and Fri; watch by Mon and have questions.
    \item Last time: MO diagram for an \ce{ML6} coordination complex whose ligands only engage in $\sigma$ interactions.
    \item Today: The case where the ligands also have $\pi$ orbitals.
    \item As per usual, we follow the procedure from Lecture 6.1.
    \begin{enumerate}
        \item Point group: $O_h$.
        \item Assigning coordinate axes is a bit more complicated here, but the following will work.
        \begin{figure}[h!]
            \centering
            \includegraphics[width=0.2\linewidth]{../ExtFiles/piOrientationsML6.png}
            \caption{\ce{ML6} $\pi$ $xyz$ coordinates.}
            \label{fig:xyzML6pi}
        \end{figure}
        \begin{itemize}
            \item We have two orthogonal $p\pi$ bonds.
            \item The arrows indicate the directional phase of the $p$ orbitals.
        \end{itemize}
        \item Let's create a representation for the 12 $p$ orbitals capable of $\pi$ bonding.
        \begin{table}[h!]
            \centering
            \small
            \renewcommand{\arraystretch}{1.2}
            \begin{tabular}{c|cccccccccc}
                $O_h$ & $E$ & $8C_3$ & $6C_2$ & $6C_4$ & $3C_2$ & $i$ & $6S_4$ & $8S_6$ & $3\sigma_h$ & $6\sigma_d$\\
                \hline
                $\Gamma_{12\pi}$ & 12 & 0 & 0 & 0 & $-4$ & 0 & 0 & 0 & 0 & 0\\
            \end{tabular}
            \caption{Representation for the $p\pi$ ligand orbitals of \ce{ML6}.}
            \label{tab:ML6pRR}
        \end{table}
        \item Reducing, we get
        \begin{equation*}
            \Gamma_{12\pi} = T_{1g}+T_{1u}+T_{2g}+T_{2u}
        \end{equation*}
        \item From Table \ref{tab:charTableOh}, the atomic orbitals of the metal transform as
        \begin{align*}
            s &\sim a_{1g}&
            p_x,p_y,p_z &\sim t_{1u}&
            d_{z^2},d_{x^2-y^2} &\sim e_g&
            d_{xz},d_{yz},d_{xy} &\sim t_{2g}
        \end{align*}
        \begin{itemize}
            \item Thus, the metal $p$ and $d_{xz},d_{yz},d_{xy}$ orbitals interact with the ligand $\pi$ SALCS.
        \end{itemize}
        \pagebreak
        \item Once again, we can "guess" the SALCs based on experience.
        \begin{figure}[H]
            \centering
            \begin{subfigure}[b]{0.24\linewidth}
                \centering
                \begin{tikzpicture}
                    \footnotesize
                    \draw
                        (0,0) coordinate (M) -- ++(90:4em)   coordinate (L1)
                        (0,0)                -- ++(-160:4em) coordinate (L2)
                        (0,0)                -- ++(-20:4em)  coordinate (L3)
                        (0,0)                -- ++(20:4em)   coordinate (L4)
                        (0,0)                -- ++(160:4em)  coordinate (L5)
                        (0,0)                -- ++(-90:4em)  coordinate (L6)
                    ;
        
                    \filldraw [semithick,fill=white] (M)
                        to[out=0,in=0,out looseness=0.5] ++(0,0.5)
                        to[out=180,in=180,in looseness=0.5] cycle
                    ;
                    \filldraw [semithick,fill=grz] (M)
                        to[out=0,in=0,out looseness=0.5] ++(0,-0.5)
                        to[out=180,in=180,in looseness=0.5] cycle
                    ;
                    \foreach \n in {L2,L3,L4,L5} {
                        \filldraw [semithick,fill=white] (\n)
                            to[out=0,in=0,out looseness=0.5] ++(0,0.3)
                            to[out=180,in=180,in looseness=0.5] cycle
                        ;
                        \filldraw [semithick,fill=grz] (\n)
                            to[out=0,in=0,out looseness=0.5] ++(0,-0.3)
                            to[out=180,in=180,in looseness=0.5] cycle
                        ;
                    }
                \end{tikzpicture}
                \caption{$t_{1u}$ bonding.}
                \label{fig:SALCML6pia}
            \end{subfigure}
            \begin{subfigure}[b]{0.24\linewidth}
                \centering
                \begin{tikzpicture}
                    \footnotesize
                    \draw
                        (0,0) coordinate (M) -- ++(90:4em)   coordinate (L1)
                        (0,0)                -- ++(-160:4em) coordinate (L2)
                        (0,0)                -- ++(-20:4em)  coordinate (L3)
                        (0,0)                -- ++(20:4em)   coordinate (L4)
                        (0,0)                -- ++(160:4em)  coordinate (L5)
                        (0,0)                -- ++(-90:4em)  coordinate (L6)
                    ;
        
                    \filldraw [semithick,fill=white,rotate=0,yscale=0.6] (M)
                        to[out=0,in=0,out looseness=0.5,in looseness=1.5] ++(0,0.65)
                        to[out=180,in=180,out looseness=1.5,in looseness=0.5] cycle
                    ;
                    \filldraw [semithick,fill=grz,rotate=-90] (M)
                        to[out=0,in=0,out looseness=0.5,in looseness=1.5] ++(0,0.65)
                        to[out=180,in=180,out looseness=1.5,in looseness=0.5] cycle
                    ;
                    \filldraw [semithick,fill=grz,rotate=90] (M)
                        to[out=0,in=0,out looseness=0.5,in looseness=1.5] ++(0,0.65)
                        to[out=180,in=180,out looseness=1.5,in looseness=0.5] cycle
                    ;
                    \filldraw [semithick,fill=white,rotate=180,yscale=0.6] (M)
                        to[out=0,in=0,out looseness=0.5,in looseness=1.5] ++(0,0.65)
                        to[out=180,in=180,out looseness=1.5,in looseness=0.5] cycle
                    ;
        
                    \foreach \n/\a in {L2/-110,L3/110,L4/70,L5/-70} {
                        \filldraw [semithick,fill=white,rotate=\a] (\n)
                            to[out=0,in=0,out looseness=0.5] ++(0,0.3)
                            to[out=180,in=180,in looseness=0.5] cycle
                        ;
                        \filldraw [semithick,fill=grz,rotate=\a] (\n)
                            to[out=0,in=0,out looseness=0.5] ++(0,-0.3)
                            to[out=180,in=180,in looseness=0.5] cycle
                        ;
                    }
                \end{tikzpicture}
                \caption{$t_{2g}$ bonding.}
                \label{fig:SALCML6pib}
            \end{subfigure}
            \begin{subfigure}[b]{0.24\linewidth}
                \centering
                \begin{tikzpicture}
                    \footnotesize
                    \draw
                        (0,0) coordinate (M) -- ++(90:4em)   coordinate (L1)
                        (0,0)                -- ++(-160:4em) coordinate (L2)
                        (0,0)                -- ++(-20:4em)  coordinate (L3)
                        (0,0)                -- ++(20:4em)   coordinate (L4)
                        (0,0)                -- ++(160:4em)  coordinate (L5)
                        (0,0)                -- ++(-90:4em)  coordinate (L6)
                    ;
        
                    \foreach \n/\a in {L2/-110,L3/-70,L4/70,L5/110} {
                        \filldraw [semithick,fill=white,rotate=\a] (\n)
                            to[out=0,in=0,out looseness=0.5] ++(0,0.3)
                            to[out=180,in=180,in looseness=0.5] cycle
                        ;
                        \filldraw [semithick,fill=grz,rotate=\a] (\n)
                            to[out=0,in=0,out looseness=0.5] ++(0,-0.3)
                            to[out=180,in=180,in looseness=0.5] cycle
                        ;
                    }
                \end{tikzpicture}
                \caption{$t_{1g}$ nonbonding.}
                \label{fig:SALCML6pic}
            \end{subfigure}
            \begin{subfigure}[b]{0.24\linewidth}
                \centering
                \begin{tikzpicture}
                    \footnotesize
                    \draw
                        (0,0) coordinate (M) -- ++(90:4em)   coordinate (L1)
                        (0,0)                -- ++(-160:4em) coordinate (L2)
                        (0,0)                -- ++(-20:4em)  coordinate (L3)
                        (0,0)                -- ++(20:4em)   coordinate (L4)
                        (0,0)                -- ++(160:4em)  coordinate (L5)
                        (0,0)                -- ++(-90:4em)  coordinate (L6)
                    ;
        
                    \foreach \n/\a in {L2/0,L3/180,L4/0,L5/180} {
                        \filldraw [semithick,fill=white,rotate=\a] (\n)
                            to[out=0,in=0,out looseness=0.5] ++(0,0.3)
                            to[out=180,in=180,in looseness=0.5] cycle
                        ;
                        \filldraw [semithick,fill=grz,rotate=\a] (\n)
                            to[out=0,in=0,out looseness=0.5] ++(0,-0.3)
                            to[out=180,in=180,in looseness=0.5] cycle
                        ;
                    }
                \end{tikzpicture}
                \caption{$t_{2u}$ nonbonding.}
                \label{fig:SALCML6pid}
            \end{subfigure}
            \caption{\ce{ML6} $\pi$ SALCs.}
            \label{fig:SALCML6pi}
        \end{figure}
        \begin{itemize}
            \item In addition to these, there are two more (oriented along the other orthogonal coordinate axes) for each IRR.
        \end{itemize}
        \item We can now make sense of the MO diagram.
        \begin{figure}[h!]
            \centering
            \includegraphics[width=0.6\linewidth]{../ExtFiles/MOsML6pi.png}
            \caption{\ce{ML6} $\pi$ MO diagram.}
            \label{fig:MOsML6pi}
        \end{figure}
        \begin{itemize}
            \item Wuttig compares this MO diagram to the all-$\sigma$ case (Figure \ref{fig:MOsML6}).
            \begin{itemize}
                \item The \ce{M}-\ce{L\sigma} and \ce{M}-\ce{L\sigma^*} distortions stay the same; we now just have \emph{additional} \ce{M}-\ce{L\pi} and \ce{M}-\ce{L\pi^*} distortions to consider.
                \item $\pi$-donating ligands, such as the fluoride ligands causing the interactions in Figure \ref{fig:MOsML6pi}, \emph{raise} the $t_{2g}$ set in energy; they contribute \emph{antibonding} interactions.
                \item Note that the \ce{L\pi} orbitals sit above the \ce{L\sigma} orbitals.
            \end{itemize}
            \item Wuttig probably expects us to be familiar with the \ce{M}-\ce{L\pi^*} notation.
        \end{itemize}
    \end{enumerate}
    \item It seems that now we're done. But wait: We have made an assumption that is not necessarily justified in every case.
    \begin{itemize}
        \item We have treated the ligand as a point particle with atomic orbitals that mirror the metal center (i.e., your typical $s$, $p$, etc. orbitals).
        \item This is justified in the case of fluoride (as in Figure \ref{fig:MOsML6pi}). But what about a ligand such as carbon monoxide? \ce{CO} certainly has molecular orbitals more complicated than the atomic orbitals of either carbon or oxygen alone, so is it still valid to treat it as a point particle with "atomic" orbitals?
        \item In fact, it is not, and we will now see how to treat that case.
    \end{itemize}
    \item As with fluoride, the frontier orbitals of \ce{CO} will be the ones that interact with the metal center.
    \item To draw SALCs of the interactions of these MOs with the metal center, we need to know what \emph{they} look like first. Fortunately, we have encountered the SALCs for heteronuclear diatomics before, and we can simply use these as our basis set to draw the overall SALCs.
    \begin{figure}[h!]
        \centering
        \begin{subfigure}[b]{0.24\linewidth}
            \centering
            \begin{tikzpicture}
                \footnotesize
                \draw
                    (0,0) coordinate (M) -- ++(90:4em)   coordinate (L1)
                    (0,0)                -- ++(-160:4em) coordinate (L2)
                    (0,0)                -- ++(-20:4em)  coordinate (L3)
                    (0,0)                -- ++(20:4em)   coordinate (L4)
                    (0,0)                -- ++(160:4em)  coordinate (L5)
                    (0,0)                -- ++(-90:4em)  coordinate (L6)
                ;
    
                \filldraw [semithick,fill=white] (M)
                    to[out=0,in=0,out looseness=0.5] ++(0,0.5)
                    to[out=180,in=180,in looseness=0.5] cycle
                ;
                \filldraw [semithick,fill=grz] (M)
                    to[out=0,in=0,out looseness=0.5] ++(0,-0.5)
                    to[out=180,in=180,in looseness=0.5] cycle
                ;
                \foreach \n in {L2,L3,L4,L5} {
                    \filldraw [semithick,fill=white] ($(M)!0.8!(\n)$)
                        to[out=0,in=0,out looseness=0.5] ++(0,0.3)
                        to[out=180,in=180,in looseness=0.5] cycle
                    ;
                    \filldraw [semithick,fill=grz] ($(M)!0.8!(\n)$)
                        to[out=0,in=0,out looseness=0.5] ++(0,-0.3)
                        to[out=180,in=180,in looseness=0.5] cycle
                    ;
                    \filldraw [semithick,fill=white] (\n)
                        to[out=0,in=0,out looseness=0.5] ++(0,-0.3)
                        to[out=180,in=180,in looseness=0.5] cycle
                    ;
                    \filldraw [semithick,fill=grz] (\n)
                        to[out=0,in=0,out looseness=0.5] ++(0,0.3)
                        to[out=180,in=180,in looseness=0.5] cycle
                    ;
                }
            \end{tikzpicture}
            \caption{$t_{1u}$ bonding.}
            \label{fig:SALCMCO6a}
        \end{subfigure}
        \begin{subfigure}[b]{0.24\linewidth}
            \centering
            \begin{tikzpicture}
                \footnotesize
                \draw
                    (0,0) coordinate (M) -- ++(90:4em)   coordinate (L1)
                    (0,0)                -- ++(-160:4em) coordinate (L2)
                    (0,0)                -- ++(-20:4em)  coordinate (L3)
                    (0,0)                -- ++(20:4em)   coordinate (L4)
                    (0,0)                -- ++(160:4em)  coordinate (L5)
                    (0,0)                -- ++(-90:4em)  coordinate (L6)
                ;
    
                \filldraw [semithick,fill=white,rotate=0,yscale=0.6] (M)
                    to[out=0,in=0,out looseness=0.5,in looseness=1.5] ++(0,0.65)
                    to[out=180,in=180,out looseness=1.5,in looseness=0.5] cycle
                ;
                \filldraw [semithick,fill=grz,rotate=-90] (M)
                    to[out=0,in=0,out looseness=0.5,in looseness=1.5] ++(0,0.65)
                    to[out=180,in=180,out looseness=1.5,in looseness=0.5] cycle
                ;
                \filldraw [semithick,fill=grz,rotate=90] (M)
                    to[out=0,in=0,out looseness=0.5,in looseness=1.5] ++(0,0.65)
                    to[out=180,in=180,out looseness=1.5,in looseness=0.5] cycle
                ;
                \filldraw [semithick,fill=white,rotate=180,yscale=0.6] (M)
                    to[out=0,in=0,out looseness=0.5,in looseness=1.5] ++(0,0.65)
                    to[out=180,in=180,out looseness=1.5,in looseness=0.5] cycle
                ;
    
                \foreach \n/\a in {L2/-110,L3/110,L4/70,L5/-70} {
                    \filldraw [semithick,fill=white,rotate=\a] ($(M)!0.8!(\n)$)
                        to[out=0,in=0,out looseness=0.5] ++(0,0.3)
                        to[out=180,in=180,in looseness=0.5] cycle
                    ;
                    \filldraw [semithick,fill=grz,rotate=\a] ($(M)!0.8!(\n)$)
                        to[out=0,in=0,out looseness=0.5] ++(0,-0.3)
                        to[out=180,in=180,in looseness=0.5] cycle
                    ;
                    \filldraw [semithick,fill=white,rotate=\a] (\n)
                        to[out=0,in=0,out looseness=0.5] ++(0,-0.3)
                        to[out=180,in=180,in looseness=0.5] cycle
                    ;
                    \filldraw [semithick,fill=grz,rotate=\a] (\n)
                        to[out=0,in=0,out looseness=0.5] ++(0,0.3)
                        to[out=180,in=180,in looseness=0.5] cycle
                    ;
                }
            \end{tikzpicture}
            \caption{$t_{2g}$ bonding.}
            \label{fig:SALCMCO6b}
        \end{subfigure}
        \caption{\ce{M(CO)6} $\pi$-bonding SALCs.}
        \label{fig:SALCMCO6}
    \end{figure}
    \begin{itemize}
        \item Wuttig draws the Figure \ref{fig:SALCMCO6a} interactions in-plane, too, though??
        \item Wuttig also doesn't draw the nonbonding ones, but they still exist??
        \item To reiterate, the $\pi^*$ orbitals of \ce{CO} will participate in $t_{1u}$ and $t_{2g}$ bonding interactions, and $t_{1g}$ and $t_{2u}$ nonbonding interactions just like the $p$ orbitals of \ce{F}; it is \emph{strictly} and \emph{solely} the basis set that we're changing.
    \end{itemize}
    \item An additional complication arises from the fact that the frontier orbitals of \ce{CO} are fundamentally different than those of \ce{F}.
    \begin{itemize}
        \item In particular, the frontier orbitals of \ce{F} are filled $\pi$-donating atomic orbitals, while \ce{CO} has a filled $\sigma$-donating frontier orbital (HOMO) and an unfilled $\pi^*$-accepting frontier orbital (LUMO).
        \item Thus, we need to consider the new \emph{energetics} of these orbitals as well.
        \item For a $\pi$-donating ligand such as monoatomic fluoride, the $\sigma$ and $\pi$ orbitals are degenerate.
        \begin{itemize}
            \item This is because all 18 ligand orbitals come from the $2p$ \emph{atomic} orbitals of fluorine.
        \end{itemize}
        \item However\dots
    \end{itemize}
    \item Are the $\sigma$ and $\pi$ orbitals of the $\pi$-accepting ligands degenerate in energy?
    \begin{itemize}
        \item They are not.
        \item This is because we are considering the interactions of nondegenerate ligand \emph{molecular} orbitals with the metal center.
        \item Evidence: We can inspect the photoelectron spectrum of our ligand (e.g., for \ce{CO}, we observe distinct peaks corresponding to its $\sigma$-donating and $\pi^*$-accepting orbitals).
        \item Note that ligands such as \ce{CO} still have filled $\pi$ MOs; it's just that these lie so low in energy that they don't interact with the metal center.
    \end{itemize}
    \item The consequence of this is that the $\pi$ ligand orbitals of a $\pi$-accepting ligand lie significantly higher in energy.
    \begin{itemize}
        \item In fact, they lie higher in energy than a metal's $d$ orbitals, meaning that the metal $t_{2g}$ set is now \ce{M}-\ce{L\pi} \emph{bonding} instead of \emph{antibonding} and hence lower in energy, leading to a greater $d$ orbital splitting.
    \end{itemize}
    \item All of this can be summarized by the MO diagram for a \ce{ML6} complex with $\pi$-accepting ligands.
    \begin{figure}[h!]
        \centering
        \includegraphics[width=0.6\linewidth]{../ExtFiles/MOsML6piAcc.png}
        \caption{\ce{ML6} $\pi$-accepting MO diagram.}
        \label{fig:MOsML6piAcc}
    \end{figure}
    \item Distortions from $\sigma$ interactions.
    \item Consider a \ce{ML6} complex with $\sigma$-only interactions, as discussed last class.
    \begin{itemize}
        \item Goal: Predict Jahn-Teller distortions from first principles.
        \item Two possible distortions: A tetragonal compression or a Jahn-Teller elongation.
        \begin{itemize}
            \item See Figure VI.10 of \textcite{bib:CHEM20100Notes}.
            \item Are both of these distortions not "Jahn-Teller" effects??
        \end{itemize}
        \item Either distortion changes the point group from $O_h$ to $D_{4h}$.
    \end{itemize}
    \item We now seek to build an MO diagram for the $t_{2g}$ and $e_g$ set of each distortion.
    \begin{table}[h!]
        \centering
        \small
        \renewcommand{\arraystretch}{1.2}
        \begin{tabular}{c|c|c}
             & $O_h$ & $D_{4h}$\\
            \hline
            $d_{x^2-y^2}$ & $e_g$ & $b_{1g}$\\
            $d_{z^2}$ & $e_g$ & $a_{1g}$\\
            $d_{xy}$ & $t_{2g}$ & $b_{2g}$\\
            $d_{yz}$ & $t_{2g}$ & $e_g$\\
            $d_{xz}$ & $t_{2g}$ & $e_g$\\
        \end{tabular}
        \caption{$d$-orbital symmetries in $O_h$ vs. $D_{4h}$.}
        \label{tab:dOrbOhD4h}
    \end{table}
    \begin{itemize}
        \item To begin, we determine the symmetries are the $d$ orbitals in the two point groups (see Table \ref{tab:dOrbOhD4h}).
        \item Since the $t_{2g}$ set is nonbonding in a $\sigma$-only complex, their energy doesn't change much, so the distorted $t_{2g}$ set is basically still degenerate. However, we do have to draw the $b_{2g}$ MO slightly higher (or lower, but we choose higher) since it has a different symmetry.
        \item The $e_g$ orbitals definitively split, going in different directions as the axial ligands compress or expand. In particular, compressing the axial ligands drives $d_{z^2}$ up (and $d_{x^2-y^2}$ down), and vice versa for expanding the axial ligands.
        \item Note that the original $e_g$ set is \ce{M}-\ce{L\sigma^*}, which is why compression drives up energy (more mixing between regions of opposite phases is not energetically favorable).
    \end{itemize}
\end{itemize}



\section{Descent in Symmetry}
\begin{itemize}
    \item \marginnote{11/9:}Goal: Consequences of distortion and descent in symmetry on the MOs of coordination complexes.
    \item \textbf{Homoleptic}: A coordination complex, the ligands of which are all identical.
    \item \textbf{Heteroleptic}: A coordination complex containing at least two distinct ligands.
    \item Most interesting chemistry occurs for heteroleptics, so we should consider their MOs, too.
    \item But how do we construct such molecular orbitals? Use a Descent in Symmetry Analysis.
    \begin{itemize}
        \item Similar to the Jahn-Teller distortion and tetragonal compression discussed in Lecture 16, but now we're taking this a step further by fully removing ligands. We then investigate the effects of this on orbital energetics.
    \end{itemize}
    \item How do the symmetries of the $d$ orbitals compare from \ce{Co(CN)6} to the hypothetical "chopped off \ce{Co(CN)5} complex," i.e., $C_{4v}$ fragment?
    \begin{itemize}
        \item We rip a cyano group off of the axial position and will substitute a bromo group later.
        \item Changing from $O_h$ to $C_{4v}$ involves a change in the symmetries of the orbitals, as follows.
        \begin{table}[h!]
            \centering
            \small
            \renewcommand{\arraystretch}{1.2}
            \begin{tabular}{c|c|c}
                 & $O_h$ & $C_{4v}$\\
                \hline
                $d_{x^2-y^2}$ & $e_g$ & $b_1$\\
                $d_{z^2}$ & $e_g$ & $a_1$\\
                $d_{xz}$ & $t_{2g}$ & $e$\\
                $d_{yz}$ & $t_{2g}$ & $e$\\
                $d_{xy}$ & $t_{2g}$ & $b_2$\\
            \end{tabular}
            \caption{$d$-orbital symmetries in $O_h$ vs. $C_{4v}$.}
            \label{tab:dOrbOhC4v}
        \end{table}
        \item Taking a $\pi$-acceptor off of the $z$-axis won't affect $d_{x^2-y^2}$; we just change the label from $e_g\mapsto b_1$.
        \item Since $d_{z^2}$ now has a less productive $\sigma$-bonding interaction, the \ce{M}-\ce{L\sigma^*} $d_{z^2}$ orbital decreases in energy (and changes to $a_1$).
        \item As with $d_{x^2-y^2}$, $d_{xy}$ doesn't change except in symmetry.
        \item Analogously to $d_{z^2}$, $d_{xz}$ and $d_{yz}$ now increase in energy because we lose the stabilizing effect of the $\pi^*$ ligand acceptance.
    \end{itemize}
    \item Now what happens when we add \ce{Br-} back in?
    \begin{itemize}
        \item \ce{Br-} is a $\pi$-donor!
        \item Quantitatively, we need to find the symmetry of the \ce{Br-} atomic orbitals and mix them with the MOs of the fragment.
        \item Since $p_z$ has $a_1$ symmetry, it can interact with the $d_{z^2}$ orbital of \ce{Co}. Thus, it will be \ce{M}-\ce{L\sigma^*} with respect to the metal, and \ce{M}-\ce{L\sigma} with respect to the \ce{Br-}.
        \item $p_x,p_y$ have $e$ symmetry. Thus, they can interact with $d_{xz},d_{yz}$ of the metal via \ce{M}-\ce{L\pi^*} interactions, raising the $e$ set higher.
    \end{itemize}
    \item Key point: Bonding character is mixed in heteroleptics because MOs have multiple parentage. Let's build the MO for the target complex now.
    \begin{itemize}
        \item The $a_1$ and $e$ sets of the \ce{Co(CN)5} fragment are both raised in energy.
        \item We see multiple parentage in the $e$ set, for instance, where we have \ce{M}-\ce{L\pi} interactions with cyano ligands and \ce{M}-\ce{L\pi^*} interactions with the bromo ligand.
        \begin{itemize}
            \item Cyano ligands act as stabilizing $\pi$-acceptors on $d_{xz},d_{yz}$.
            \item Bromo ligands act as destabilizing $\pi$-donors on $d_{xz},d_{yz}$.
        \end{itemize}
        \item We fill in 6 $d$ electrons for \ce{[Co(CN)5Br]^3-} because such a structure implies \ce{Co^3+}, which has 6 $d$ electrons.
    \end{itemize}
    \item Example: \ce{NbCl5O}.
    \begin{figure}[h!]
        \centering
        \begin{tikzpicture}
            \footnotesize
            \begin{scope}
                \draw [ultra thick]
                    (-0.55,1) -- ++(0.5,0) ++(0.1,0) -- ++(0.5,0) coordinate (1egr)
                    (-0.85,0) -- ++(0.5,0) ++(0.1,0) -- ++(0.5,0) ++(0.1,0) -- ++(0.5,0) coordinate (1t2gr)
                ;
                \node [left] at (-0.85,1) {\ce{M}-\ce{L\sigma^*} $e_g$};
                \node [left] at (-0.85,0) {\ce{M}-\ce{L\pi^*} $t_{2g}$};
            \end{scope}
            \begin{scope}[xshift=3cm]
                \draw [ultra thick]
                    (-0.25,1) coordinate (2b1l)   -- ++(0.5,0) coordinate (2b1r)
                    (-0.25,0.7) coordinate (2a1l) -- ++(0.5,0) coordinate (2a1r)
                    (-0.25,0) coordinate (2b2l)   -- ++(0.5,0) coordinate (2b2r)
                    (-0.55,-0.3) coordinate (2el) -- ++(0.5,0) ++(0.1,0) -- ++(0.5,0) coordinate (2er)
                ;
                \node [above] at (0,1) {$b_1$};
                \node [below] at (0,0.7) {$a_1$};
                \node [above] at (0,0) {$b_2$};
                \node [below] at (0,-0.3) {$e$};
            \end{scope}
            \begin{scope}[xshift=6cm]
                \draw [ultra thick]
                    (-0.25,1) coordinate (3b1l)   -- ++(0.5,0)
                    (-0.25,0.8) coordinate (3a1l) -- ++(0.5,0)
                    (-0.25,0) coordinate (3b2l)   -- ++(0.5,0)
                    (-0.55,-0.2) coordinate (3el) -- ++(0.5,0) ++(0.1,0) -- ++(0.5,0)
                ;
                \node [right] at (0.55,1.05) {$b_1$};
                \node [right] at (0.55,0.75) {$a_1$};
                \node [right] at (0.55,0.05) {$b_2$};
                \node [right] at (0.55,-0.25) {$e$};
            \end{scope}
            \draw [grx,thick,densely dashed]
                (1egr)  -- (2b1l)
                (1egr)  -- (2a1l)
                (1t2gr) -- (2b2l)
                (1t2gr) -- (2el)
                (2b1r)  -- (3b1l)
                (2a1r)  -- (3a1l)
                (2b2r)  -- (3b2l)
                (2er)   -- (3el)
            ;
        \end{tikzpicture}
        \caption{\ce{NbCl5O} $d$ orbitals derivation.}
        \label{fig:dOrbsNbCl5O}
    \end{figure}
    \begin{itemize}
        \item Remove a $\pi$-donor; add back another $\pi$-donor.
        \item Thus, we drop the $e$ and $a_1$ sets, and then raise them back up (but not all the way to degeneracy).
    \end{itemize}
    \item The \ce{M}-\ce{L\sigma^*} notation denotes molecular orbital \textbf{parentage}!
    \item Further descents in symmetry: Two ligands gone means $D_{4h}$; three ligands gone means $C_{2v}$.
    \begin{figure}[H]
        \centering
        \begin{tikzpicture}[
            text height=1.5ex,text depth=0.25ex
        ]
            \footnotesize
            \begin{scope}
                \draw [ultra thick]
                    (-0.55,2) -- ++(0.5,0) ++(0.1,0) -- ++(0.5,0) coordinate (1egr)
                    (-0.85,0) -- ++(0.5,0) ++(0.1,0) -- ++(0.5,0) ++(0.1,0) -- ++(0.5,0) coordinate (1t2gr)
                ;
                \node [left] at (-0.85,2) {\ce{M}-\ce{L\sigma^*} $e_g$};
                \node [left] at (-0.85,0) {nb $t_{2g}$};
    
                \node at (0,-1) {\small$O_h$};
    
                \draw
                    (0,4) -- ++(20:0.5)
                    (0,4) -- ++(90:0.5)
                    (0,4) -- ++(160:0.5)
                    (0,4) -- ++(-160:0.5)
                    (0,4) -- ++(-90:0.5)
                    (0,4) -- ++(-20:0.5)
                ;
            \end{scope}
            \begin{scope}[xshift=3cm]
                \draw [ultra thick]
                    (-0.25,2)     coordinate (2b1l) -- ++(0.5,0) coordinate (2b1r)
                    (-0.25,1.7)   coordinate (2a1l) -- ++(0.5,0) coordinate (2a1r)
                    (-0.55,0.05)  coordinate (2el)  -- ++(0.5,0) ++(0.1,0) -- ++(0.5,0) coordinate (2er)
                    (-0.25,-0.05) coordinate (2b2l) -- ++(0.5,0) coordinate (2b2r)
                ;
                \node [above] at (0,2) {$b_1$};
                \node [below] at (0,1.7) {$a_1$};
                \node [above] at (0,0.05) {$e$};
                \node [below] at (0,-0.05) {$b_2$};
    
                \node at (0,-1) {\small$C_{4v}$};
    
                \draw
                    (0,4) -- ++(20:0.5)
                    (0,4) -- ++(160:0.5)
                    (0,4) -- ++(-160:0.5)
                    (0,4) -- ++(-90:0.5)
                    (0,4) -- ++(-20:0.5)
                ;
            \end{scope}
            \begin{scope}[xshift=6cm]
                \draw [ultra thick]
                    (-0.25,2)     coordinate (3b1gl) -- ++(0.5,0) coordinate (3b1gr)
                    (-0.25,1.4)   coordinate (3a1gl) -- ++(0.5,0) coordinate (3a1gr)
                    (-0.55,0.05)  coordinate (3egl)  -- ++(0.5,0) ++(0.1,0) -- ++(0.5,0) coordinate (3egr)
                    (-0.25,-0.05) coordinate (3b2gl) -- ++(0.5,0) coordinate (3b2gr)
                ;
                \node [above] at (0,2) {$b_{1g}$};
                \node [below] at (0,1.4) {$a_{1g}$};
                \node [above] at (0,0.05) {$e_g$};
                \node [below] at (0,-0.05) {$b_{2g}$};
    
                \node at (0,-1) {\small$D_{4h}$};
    
                \draw
                    (0,4) -- ++(20:0.5)
                    (0,4) -- ++(160:0.5)
                    (0,4) -- ++(-160:0.5)
                    (0,4) -- ++(-20:0.5)
                ;
            \end{scope}
            \begin{scope}[xshift=9cm]
                \draw [ultra thick]
                    (-0.25,1.8)  coordinate (4b1lt) -- ++(0.5,0)
                    (-0.25,1.2)  coordinate (4a1l)  -- ++(0.5,0)
                    (-0.25,0.1)  coordinate (4a2l)  -- ++(0.5,0)
                    (-0.25,0)    coordinate (4b2l)  -- ++(0.5,0)
                    (-0.25,-0.1) coordinate (4b1lb) -- ++(0.5,0)
                ;
                \node [right] at (0.25,1.8)  {$b_1$};
                \node [right] at (0.25,1.2)  {$a_1$};
                \node [right] at (0.25,0.3)  {$a_2$};
                \node [right] at (0.25,0)    {$b_2$};
                \node [right] at (0.25,-0.3) {$b_1$};
    
                \node at (0,-1) {\small$C_{2v}$};
    
                \draw
                    (0,4) -- ++(20:0.5)
                    (0,4) -- ++(160:0.5)
                    (0,4) -- ++(-160:0.5)
                ;
            \end{scope}
            \draw [grx,thick,densely dashed]
                (1egr)  -- (2b1l)
                (1egr)  -- (2a1l)
                (1t2gr) -- (2el)
                (1t2gr) -- (2b2l)
                (2b1r)  -- (3b1gl)
                (2a1r)  -- (3a1gl)
                (2er)   -- (3egl)
                (2b2r)  -- (3b2gl)
                (3b1gr) -- (4b1lt)
                (3a1gr) -- (4a1l)
                (3egr)  -- (4a2l)
                (3egr)  -- (4b2l)
                (3b2gr) -- (4b1lb)
            ;
        \end{tikzpicture}
        \caption{Full descent in symmetry.}
        \label{fig:descentSymmetry}
    \end{figure}
    \begin{itemize}
        \item Wuttig considers the $\sigma$-only case here.
        \item $D_{4h}$: Another $z$ $\sigma$-donor gone lowers $a_1\mapsto a_{1g}$ further.
        \item $C_{2v}$: Fewer $z$ donors means lower $a_1$. Fewer $xy$-donors means lower $b_1$. The nonbonding $t_{2g}$ set is still unchanged.
    \end{itemize}
\end{itemize}



\section{Metal-Metal Bonding}
\begin{itemize}
    \item \marginnote{11/11:}Picking up from last time with Walsh diagrams; we'll get to metal-metal bonding later.
    \item Last time, we talked about descents in symmetry from one particular point group to another destination point group.
    \item Today: What about intermediate geometries that don't quite fit any particular point group?
    \begin{itemize}
        \item We use \textbf{Walsh diagrams} to treat these cases.
    \end{itemize}
    \item \textbf{Walsh diagram}: A plot of orbital energy vs. a measure of distortion, often the angle $\theta$ between some bonds.
    \item Let's look at an example of a Walsh diagram. We'll break down how and when to build one, and what information we can glean from one.
    \item Consider a square pyramidal $C_{4v}$ complex.
    \begin{figure}[h!]
        \centering
        \includegraphics[width=0.5\linewidth]{../ExtFiles/WalshC4v.png}
        \caption{Walsh diagram for the distortion of a $C_{4v}$ coordination complex.}
        \label{fig:WalshC4v}
    \end{figure}
    \begin{itemize}
        \item As we increase the angle between the axial ligand and each equatorial ligand, we remain in the $C_{4v}$ point group, but we have changes among orbital energy, too.
        \item Derive one side, then the other side, and then draw a line connecting the two analogous orbitals.
    \end{itemize}
    \item A benefit of Walsh diagrams: They allow us to predict the distortion as a function of electron count.
    \item Continuing with the $C_{4v}$ example in Figure \ref{fig:WalshC4v}\dots
    \begin{itemize}
        \item For $d^0$-$d^2$, steric pressures imply a distorted pyramid; electronic stability isn't significant.
        \item For $d^3$-$d^6$, however, we can expect a flat pyramid since the orbitals that are now being filled ($d_{xz},d_{yz}$ of $e$ symmetry) are destabilized in the distorted geometry. In fact, the more electrons we add, the stronger the preference for flat.
        \item For $d^7$-$d^8$, the equilibrium shifts back toward distortion ($x^2-y^2$ gets stabilized by distortion).
        \item For $d^9$-$d^{10}$, the distortion grows even more rapidly ($z^2$ also gets stabilized by distortion).
    \end{itemize}
    \item What happens when the principal rotation axis is not preserved as a function of descending symmetry?
    \item Consider the case where we go from distorted $C_{4v}$ to trigonal bipyramidal ($D_{3h}$).
    \begin{figure}[h!]
        \centering
        \includegraphics[width=0.5\linewidth]{../ExtFiles/DistortC4vD3h.png}
        \caption{Distortion from $C_{4v}$ to $D_{3h}$.}
        \label{fig:DistortC4vD3h}
    \end{figure}
    \begin{itemize}
        \item In a nutshell, we must derive $d$-orbital manifolds in both point groups and connect them just like before, but we now have the additional challenge of orbital labels changing. Let's begin.
        \item Note that not only does the point group change here, but we are forced to reorient the axes if we want to keep the $z$-axis as the principal axis.
        \item Once again, we assume $\sigma$-only interactions.
        \item Determine what axis maps to what other axis: $x\mapsto x$, $z\mapsto y$, and $y\mapsto z$.
        \begin{itemize}
            \item Note that we don't send $y\mapsto -z$ because we are only interested in direction (mapping \emph{axes} to \emph{axes}), not intra-basis orientation.
        \end{itemize}
    \end{itemize}
    \item Drawing a Walsh diagram for the case we've been discussing.
    \begin{figure}[h!]
        \centering
        \begin{tikzpicture}[
            text height=1.5ex,text depth=0.25ex
        ]
            \footnotesize
            \begin{scope}
                \draw [ultra thick]
                    (-0.25,2)     -- node[above]{$d_{x^2-y^2}$} ++(0.5,0) coordinate (1b1r)
                    (-0.25,1.7)   -- node[below]{$d_{z^2}$} ++(0.5,0) coordinate (1a1r)
                    (-0.55,0.05)  -- node[above]{$d_{xz}$} ++(0.5,0) ++(0.1,0) -- node[above]{$d_{yz}$} ++(0.5,0) coordinate (1er)
                    (-0.25,-0.05) -- node[below]{$d_{xy}$} ++(0.5,0) coordinate (1b2r)
                ;
                \node [left] at (-0.55,2)    {$b_1$ \ce{M}-\ce{L\sigma^*}};
                \node [left] at (-0.55,1.7)  {$a_1$ \ce{M}-\ce{L\sigma^*}};
                \node [left] at (-0.55,0.15)  {$e$ nb};
                \node [left] at (-0.55,-0.15) {$b_2$ nb};
    
                \node at (0,-1) {\small$C_{4v}$};
            \end{scope}
            \begin{scope}[xshift=3cm]
                \draw [ultra thick]
                    (-0.25,2) coordinate (2a1l)  -- node[above]{$d_{z^2}$} ++(0.5,0)
                    (-0.55,1) coordinate (2e'l)  -- node[above]{$d_{xy}$}  ++(0.5,0) ++(0.1,0) -- node[above,xshift=1mm]{$d_{x^2-y^2}$} ++(0.5,0)
                    (-0.55,0) coordinate (2e''l) -- node[above]{$d_{xz}$}  ++(0.5,0) ++(0.1,0) -- node[above]{$d_{yz}$} ++(0.5,0)
                ;
                \node [right] at (0.7,2) {\ce{M}-\ce{L\sigma^*} $a_1'$};
                \node [right] at (0.7,1) {\ce{M}-\ce{L\sigma^*} $e'$};
                \node [right] at (0.7,0) {nb $e''$};
    
                \node at (0,-1) {\small$D_{3h}$};
            \end{scope}
            \draw [grx,thick,densely dashed]
                (1b1r) -- (2a1l)
                (1a1r) -- (2e'l)
                (1er)  -- (2e'l)
                (1er)  -- (2e''l)
                (1b2r) -- (2e''l)
            ;
        \end{tikzpicture}
        \caption{Walsh diagram when the principal axis is not preserved.}
        \label{fig:WalshNoPrincipal}
    \end{figure}
    \begin{itemize}
        \item Start with the $C_{4v}$ $d$-orbital manifold on the left and the $D_{3h}$ $d$-orbital manifold on the right.
        \item Be cognizant of the fact that we have to relabel orbitals because we relabeled axes! Indeed, $d_{xy}\mapsto d_{xz}$, $d_{yz}\mapsto d_{yz}$, and $d_{xz}\mapsto d_{xy}$.
        \begin{itemize}
            \item Thus, what we call the $d_{xz}$ orbital on the right (for example) is literally the $d_{xy}$ orbital on the left.
        \end{itemize}
        \item Wuttig rationalizes the $D_{3h}$ splitting using the ligand SALCs and their symmetries.
        \begin{itemize}
            \item For the $D_{3h}$ molecule, the ligand SALCS transform as
            \begin{table}[H]
                \centering
                \small
                \renewcommand{\arraystretch}{1.2}
                \begin{tabular}{c|cccccc}
                    $D_{3h}$ & $E$ & $2C_3$ & $3C_2'$ & $\sigma_h$ & $2S_3$ & $3\sigma_v$\\
                    \hline
                    $\Gamma_{\ce{L}\text{-}\sigma}$ & 5 & 2 & 1 & 3 & 0 & 3\\
                \end{tabular}
                \caption{Representations for the \ce{L\sigma} orbitals of a $D_{3h}$ complex.}
                \label{tab:D3hRR}
            \end{table}
            i.e.,
            \begin{equation*}
                \Gamma_{\ce{L}\text{-}\sigma} = 2a_1'+e'+a_2''
            \end{equation*}
            \item The $a_2''$ SALCs are nonbonding, but $e'$ and $a_1'$ both have \ce{M}-\ce{L\sigma^*} interactions.
            \item How did Wuttig determine which of $d_{x^2-y^2}$ and $d_{z^2}$ went to which product orbitals??
        \end{itemize}
        \item Connect like orbitals with lines.
    \end{itemize}
    \item Using MO theory to predict if complexes containing \ce{NO} as a ligand prefer bent or linear \ce{NO}.
    \begin{figure}[H]
        \centering
        \footnotesize
        \chemfig{Ir(-[:-10]PPh_3)(<:[:-70]Cl)(<[:-110]Cl)(-[:-170]Ph_3P)(-[2]N-[:30]O)}
        \caption{An \ce{NO} complex.}
        \label{fig:NOcomplex}
    \end{figure}
    \item Strategy.
    \begin{enumerate}
        \item Pyramidalize $D_{4h}$: Derive the MO diagram for just the \ce{Ir(PPh3)2Cl2} ligands, assuming all ligands are the same.
        \item Combine with \ce{NO} orbitals.
        \item Bend \ce{NO}.
    \end{enumerate}
    \item Completing the example in Figure \ref{fig:NOcomplex}.
    \begin{figure}[h!]
        \centering
        \begin{tikzpicture}[
            text height=1.5ex,text depth=0.25ex
        ]
            \footnotesize
            \begin{scope}
                \draw [ultra thick]
                    (-0.25,2) -- ++(0.5,0) coordinate (1b1gr)
                    (-0.25,1) -- ++(0.5,0) coordinate (1a1gr)
                    (-0.25,0.1) -- ++(0.5,0)
                    (-0.25,0) -- ++(0.5,0) coordinate (1egb2gr)
                    (-0.25,-0.1) -- ++(0.5,0)
                ;
                \node [above] at (0,2) {$x^2-y^2$};
                \node [above] at (0,1) {$z^2$};
                \node [above] at (0,0.1) {$xz,yz$};
                \node [below] at (0,-0.1) {$xy$};
    
                \node [left] at (-0.25,2) {$b_{1g}$};
                \node [left] at (-0.25,1) {$a_{1g}$};
                \node [left] at (-0.25,0.2) {$e_g$};
                \node [left] at (-0.25,-0.2) {$b_{2g}$};
    
                \node at (0,-2) {\small$D_{4h}$};
    
                \node at (0,4) {
                    \chemfig{M(-)(<:[1])(-[4])(<[5])}
                };
            \end{scope}
            \begin{scope}[xshift=3cm]
                \draw [ultra thick]
                    (-0.25,1.6) coordinate (2x2y2l) -- ++(0.5,0) coordinate (2x2y2r)
                    (-0.55,0.9) coordinate (2xzyzl) -- ++(0.5,0) ++(0.1,0) -- ++(0.5,0) coordinate (2xzyzr)
                    (-0.25,0.4) coordinate (2z2l) -- ++(0.5,0) coordinate (2z2r)
                    (-0.25,0) coordinate (2xyl) -- ++(0.5,0) coordinate (2xyr)
                ;
                \node [above] at (0,1.6) {$x^2-y^2$};
                \node [above] at (-0.3,0.9) {$xz$};
                \node [above] at (0.3,0.9) {$yz$};
                \node [above] at (0,0.4) {$z^2$};
                \node [above] at (0,0) {$xy$};
    
                \node at (0,4) {
                    \chemfig{M(-[:-10])(<:[:-70])(<[:-110])(-[:-170])}
                };
            \end{scope}
            \begin{scope}[xshift=6cm]
                \draw [ultra thick]
                    (-0.25,1.6) coordinate (3x2y2l) -- ++(0.5,0) coordinate (3x2y2r)
                    (-0.55,1.2) coordinate (3xz*l)  -- ++(0.5,0) ++(0.1,0) -- ++(0.5,0) coordinate (3xz*r)
                    (-0.25,0.7) coordinate (3z2ml)  -- ++(0.5,0) coordinate (3z2mr)
                    (-0.25,0)   coordinate (3xyl)   -- ++(0.5,0) coordinate (3xyr)
                    (-0.55,-0.5) coordinate (3xzl)  -- ++(0.5,0) ++(0.1,0) -- ++(0.5,0) coordinate (3xzr)
                    (-0.25,-1.2) coordinate (3z2pl) -- ++(0.5,0) coordinate (3z2pr)
                ;
                \node [above] at (0,1.6) {$x^2-y^2$};
                \node [above] at (0,1.2) {$xz,y+\pi^*$};
                \node [above] at (0,0.7) {$z^2-\lambda n$};
                \node [above] at (0,0)   {$xy$};
                \node [above] at (0,-0.5) {$xz,y+\pi$};
                \node [above] at (0,-1.2) {$n+\lambda z^2$};
    
                \node at (0,4) {
                    \chemfig{M(-[:-10])(<:[:-70])(<[:-110])(-[:-170])(-[2]N-[2]O)}
                };
            \end{scope}
            \begin{scope}[xshift=9cm]
                \draw [ultra thick]
                    (-0.25,1.1) -- ++(0.5,0)
                    (-0.25,1) -- ++(0.5,0)
                    (-0.25,-0.8) coordinate (4sigmal) -- ++(0.5,0)
                ;
                \coordinate (4pi*l) at (-0.25,1.05);
                \node [right] at (0.25,1.05) {$\pi^*$};
                \node [right] at (0.25,-0.8) {$\sigma$};
            \end{scope}
            \draw [grx,thick,densely dashed]
                (1b1gr) -- (2x2y2l)
                (1a1gr) -- (2z2l)
                (1egb2gr) -- (2xzyzl)
                (1egb2gr) -- (2xyl)
                (2x2y2r) -- (3x2y2l)
                (2xzyzr) -- (3xz*l)
                (2xzyzr) -- (3xzl)
                (2z2r) -- (3z2ml)
                (2z2r) -- (3z2pl)
                (2xyr) -- (3xyl)
                (3xz*r) -- (4pi*l)
                (3xzr) -- (4pi*l)
                (3z2mr) -- (4sigmal)
                (3z2pr) -- (4sigmal)
            ;

            \path (6,0) -- ++(0,5.7);
        \end{tikzpicture}
        \caption{\ce{Ir(PPh3)2Cl2(NO)} MOs.}
        \label{fig:MOsNOcomplex}
    \end{figure}
    \begin{itemize}
        \item Generalize all ligands to a generic $\pi$-donor (like \ce{Cl-}).
        \item Pyramidalize the molecular fragment to the $D_{4h}$ molecular geometry.
        \item Give the $d$-orbital splitting for this point group.
        \item Bend the molecule and change the orbital energies accordingly: Loss of planar antibonding interactions lowers $d_{x^2-y^2}$ and $d_{z^2}$, loss of $\sigma_h$ increases $d_{xz},d_{yz}$, no change in symmetry about the principal axis means that $d_{xy}$ is unaffected.
        \item These orbitals can be mixed with the linear \ce{NO} frontier orbitals (a high-lying $\pi^*$ one and a low-lying $\sigma$ one) to generate an initial MO diagram.
    \end{itemize}
    \item We can use a Walsh diagram to describe the distortion in the MO diagram in Figure \ref{fig:MOsNOcomplex} as we bend the \ce{NO} ligand.
    \begin{figure}[h!]
        \centering
        \includegraphics[width=0.23\linewidth]{../ExtFiles/WalshNOcomplex.png}
        \caption{Walsh diagram for varying \ce{Ir}-\ce{N}-\ce{O} bond angles.}
        \label{fig:WalshNOcomplex}
    \end{figure}
    \item Do $d^6$, $d^7$, and $d^8$ prefer linear or bent geometries?
    \begin{itemize}
        \item $d^6$ prefers linear since those orbitals lie lower in energy.
        \item $d^7$ prefers bent for the same reason (we now have an electron in the drastically changing $z^2-\lambda n$).
        \item $d^8$ prefers an intermediate angle, even though we would expect fully bent based on the Walsh diagram.
    \end{itemize}
    \item We now move to metal-metal bonding.
    \item Metal-metal bonding involves $\delta$ bonds.
    \begin{itemize}
        \item Different types of $d$-$d$ molecular orbitals are shown (see Figure III.8 of \textcite{bib:CHEM20100Notes}).
    \end{itemize}
    \item Seminal work of Al Cotton, a giant of inorganic chemistry: One of the first quadruple-bonded complexes, \ce{[Re2Cl8]^2-}.
    \item Investigating the bond-order for the \ce{Re-Re} bond.
    \begin{figure}[H]
        \centering
        \begin{tikzpicture}[
            yscale=1.5
        ]
            \footnotesize
            \draw [gry,ultra thick]
                (-2,3) coordinate (x2y2r) -- ++(-0.5,0)
                (-2,2) coordinate (z2r)   -- ++(-0.5,0)
                (-2,1) coordinate (xyr)   -- ++(-0.5,0)
                (-2,0.033)                 -- ++(-0.5,0)
                (-2,-0.033)                -- ++(-0.5,0)
            ;
            \coordinate (xzyzr) at (-2,0);
            \node [left] at (-2.5,3) {$x^2-y^2$};
            \node [left] at (-2.5,2) {$z^2$};
            \node [left] at (-2.5,1) {$xy$};
            \node [left] at (-2.5,0) {$xz,yz$};
    
            \draw [gry,ultra thick]
                (2,3) coordinate (x2y2l) -- ++(0.5,0)
                (2,2) coordinate (z2l)   -- ++(0.5,0)
                (2,1) coordinate (xyl)   -- ++(0.5,0)
                (2,0.033)                 -- ++(0.5,0)
                (2,-0.033)                -- ++(0.5,0)
            ;
            \coordinate (xzyzl) at (2,0);
            \node [right] at (2.5,3) {$x^2-y^2$};
            \node [right] at (2.5,2) {$z^2$};
            \node [right] at (2.5,1) {$xy$};
            \node [right] at (2.5,0) {$xz,yz$};
    
            \draw [ultra thick]
                (-0.25,3.8)  coordinate (s*l)  -- ++(0.5,0) coordinate (s*r)
                (-0.25,3.1)  coordinate (d*2l) -- ++(0.5,0) coordinate (d*2r)
                (-0.25,2.2)  coordinate (d2l)  -- ++(0.5,0) coordinate (d2r)
                (-0.55,1.5)  coordinate (p*l)  -- ++(0.5,0) ++(0.1,0) -- ++(0.5,0) coordinate (p*r)
                (-0.25,1.1)  coordinate (d*l)  -- ++(0.5,0) coordinate (d*r)
                (-0.25,0.2)    coordinate (dl)   -- node{\Large$\upharpoonleft\hspace{-1mm}\downharpoonright$} ++(0.5,0) coordinate (dr)
                (-0.25,-0.6) coordinate (sl)   -- node{\Large$\upharpoonleft\hspace{-1mm}\downharpoonright$} ++(0.5,0) coordinate (sr)
                (-0.55,-1.5) coordinate (pl)   -- node{\Large$\upharpoonleft\hspace{-1mm}\downharpoonright$} ++(0.5,0) ++(0.1,0) -- node{\Large$\upharpoonleft\hspace{-1mm}\downharpoonright$} ++(0.5,0) coordinate (pr)
            ;
            \node [below]     at (0,3.8)  {$\sigma^*$};
            \node [below]     at (0,3.1)  {$\delta^*$};
            \node [below]     at (0,2.2)  {$\delta$};
            \node [below]     at (0,1.5)  {$\pi^*$};
            \node [below]     at (0,1.1)  {$\delta^*$};
            \node [below=2mm] at (0,0.2)  {$\delta$};
            \node [below=2mm] at (0,-0.6) {$\sigma$};
            \node [below=2mm] at (0,-1.5) {$\pi$};
            
            \draw [grx,thick,densely dashed]
                (x2y2r) -- (d*2l)
                (x2y2r) -- (d2l)
                (z2r) -- (s*l)
                (z2r) -- (sl)
                (xyr) -- (d*l)
                (xyr) -- (dl)
                (xzyzr) -- (p*l)
                (xzyzr) -- (pl)
                (x2y2l) -- (d*2r)
                (x2y2l) -- (d2r)
                (z2l) -- (s*r)
                (z2l) -- (sr)
                (xyl) -- (d*r)
                (xyl) -- (dr)
                (xzyzl) -- (p*r)
                (xzyzl) -- (pr)
            ;
        \end{tikzpicture}
        \caption{\ce{[Re2Cl8]^2-} MOs.}
        \label{fig:MOsRe2Cl8}
    \end{figure}
    \begin{itemize}
        \item Denote the \ce{Re-Re} axis as the $z$-axis.
        \item We form two $C_{4v}$ \ce{ReCl4} fragments. Their $d$-orbital splitting (we only care about the $d$-orbitals of the metal centers since we're trying to investigate the \ce{Re-Re} bond) mirrors that in Figure \ref{fig:dOrbsNbCl5O}.
        \begin{itemize}
            \item Note that since the $d_{xy},d_{xz},d_{yz}$ orbitals are \ce{M-L\pi^*}, they are less antibonding than in the octahedral case because we no longer have the $\pi^*$ contribution from the orbitals along the $z$-axis??
        \end{itemize}
        \item We then mix their identical orbitals.
        \item Filling in the $d$-electrons on the two \ce{Re^3+} species, we see that $\text{B.O.}=4$.
    \end{itemize}
    \item $\sigma>\pi>\delta$ in terms of bond strength and MO splitting.
\end{itemize}




\end{document}