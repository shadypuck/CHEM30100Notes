\documentclass[../notes.tex]{subfiles}

\pagestyle{main}
\renewcommand{\chaptermark}[1]{\markboth{\chaptername\ \thechapter\ (#1)}{}}
\setcounter{chapter}{3}

\begin{document}




\chapter{Spectroscopy}
\section{IR Selection Rules and Stretching Mode Analysis}
\begin{itemize}
    \item \marginnote{10/17:}Fill out the Google Form to indicate topics we want Wuttig to cover during the review session.
    \item The most common experiments we do to determine normal modes are IR and Raman experiments.
    \begin{itemize}
        \item Vibration modes can be IR and/or Raman active.
        \item IR spectroscopy probes direct absorption of IR light to excite vibrational modes.
        \item We will first determine the IR selection rules.
    \end{itemize}
    \item IR selection rules:
    \begin{itemize}
        \item We want $\Delta v=\pm 1$.
        \item The transition moment integral is for transitions $v\to v'$; it is written as
        \begin{equation*}
            M_{vv'} = \int_{-\infty}^\infty\Psi^*(v')\mu\Psi(v)\dd{x}
        \end{equation*}
        where $\mu$ is the electric dipole moment and $[M_{vv'}]^2$ is the probability of the transition.
    \end{itemize}
    \item If $\mu$ were a constant, then
    \begin{equation*}
        M_{vv'} = \mu\int_{-\infty}^\infty\Psi^*(v')\Psi(v)\dd{x} = 0
    \end{equation*}
    since $\Psi^*(v')$ and $\Psi^*(v)$ are orthogonal functions.
    \begin{itemize}
        \item Therefore, $\mu$ cannot be a constant; it needs to be a function of $x$ and needs to change during the vibration for the transition to be allowed.
    \end{itemize}
    \item A more general form of $[M_{vv'}]^2$ is
    \begin{equation*}
        [M_{vv'}]^2 = \int_\text{all space}\Psi^*(v')\hat{\mu}\Psi(v)\dd\tau
    \end{equation*}
    \begin{itemize}
        \item In order for the above integral to not evaluate to zero, the direct product of the excited state wave function, transition dipole moment, and ground state wave function must contain the totally symmetric IRR. Symbolically,
        \begin{equation*}
            \Gamma_\text{IRR}(\Psi(v'))\times\Gamma_\text{IRR}(\hat{\mu})\times\Gamma_\text{IRR}(\Psi(v))
        \end{equation*}
        decomposes into a sum of IRRs including $A_1$.
    \end{itemize}
    \item Bottom line: A vibration will be IR active if it causes a change in the electric dipole moment of a molecule. A fundamental mode will be IR active if the normal mode which is excited belongs to the same representation as any one or several of the Cartesian coordinates.
    \item What modes are IR active for water?
    \begin{itemize}
        \item Recall that the vibrational modes for \ce{H2O} are $a_1$ corresponding to $\nu_a$, $b_2$ corresponding to $\nu_{as}$, and $a_1$ corresponding to $\delta$.
        \begin{itemize}
            \item Looking at the character table, we notice that both $a_1,b_2$ transform as a linear function ($z,y$, respectively), so all modes are IR active.
            \item More specifically, let's look at $b_2$. If $b_2$ transforms as a linear function, it \emph{is} true that it is IR active. Here's why: If $b_2$ transforms as a linear function, then it will be a component of $\hat{\mu}$. In this case, when we take the direct product $\Gamma_{v'}\times\hat{\mu}$, one term we will evaluate is $b_2\times b_2$. But by the second of the three important theorems, the direct product of any representation times itself will contain the totally symmetric irreducible representation, which is required for IR visibility as per the third of the three important theorems.
        \end{itemize}
        \item Show by direct product analysis that these IR modes are allowed.
        \begin{itemize}
            \item The first transition ($\nu_a$) goes from $a_1\to a_1$ (the ground state is relaxed, hence $a_1$, and the excited state is $\Gamma_\text{vibs}$ for $\nu_a$, which is $a_1$). It follows that
            \begin{equation*}
                \Psi^*(v')\hat{\mu}\Psi(v) \sim\footnote{$\sim$ denotes "transforms as."} a_1
                \begin{pmatrix}
                    a_1\\
                    b_1\\
                    b_2\\
                \end{pmatrix}
                a_1 =
                \begin{pmatrix}
                    a_1\\
                    b_1\\
                    b_2\\
                \end{pmatrix}
            \end{equation*}
            Since the result of our calculation contains the totally symmetric representation (in its first entry), we know that $v_a$ is allowed.
            \item The asymmetric stretch has ground $a_1$ and excited state $b_2$. $\hat{\mu}$ is the same as before.
            \begin{equation*}
                \Psi^*(v')\hat{\mu}\Psi(v) \sim b_2
                \begin{pmatrix}
                    a_1\\
                    b_1\\
                    b_2\\
                \end{pmatrix}
                a_1 =
                \begin{pmatrix}
                    b_2\\
                    a_2\\
                    a_1\\
                \end{pmatrix}
            \end{equation*}
            Since it still contains the all-symmetric wavefunction, it's allowed.
        \end{itemize}
    \end{itemize}
    \item Example:
    \begin{figure}[h!]
        \centering
        \begin{subfigure}[b]{0.2\linewidth}
            \centering
            \includegraphics[width=0.8\linewidth]{../ExtFiles/ML3CO3a.png}
            \caption{\emph{mer} isomer.}
            \label{fig:ML3CO3a}
        \end{subfigure}
        \begin{subfigure}[b]{0.2\linewidth}
            \centering
            \includegraphics[width=0.63\linewidth]{../ExtFiles/ML3CO3b.png}
            \caption{\emph{fac} isomer.}
            \label{fig:ML3CO3b}
        \end{subfigure}
        \caption{Isomers of an octahedral \ce{ML3(CO)3} complex.}
        \label{fig:ML3CO3}
    \end{figure}
    \begin{enumerate}
        \item Determine the number, symmetries, and IR activities of the carbonyl stretching modes for the two isomers of an octahedral \ce{ML3(CO)3} complex.
        \begin{itemize}
            \item Since we are asked to determine the \emph{carbonyl stretching} modes, we choose as our basis set the three vectors which run parallel to the \ce{CO} bonds in both cases.
            \item The point group of the \emph{mer} isomer is $C_{2v}$; the point group of the \emph{fac} isomer is $C_{3v}$.
            \item With this information, the rest of the question is fairly straightforward.
        \end{itemize}
        \item The IR spectrum of the compound \ce{Mo(CO)3[P(OCH3)3]3} exhibits bands at \qtylist{1993;1919;1890}{\per\centi\meter}. The IR spectrum of compound \ce{Cr(CO)3(CHCH3)3} exhibits bands at \qtylist{1942;1860}{\per\centi\meter}. Based on your answer from part (1), how would you assign the \emph{fac} vs. \emph{mer} structure of these two complexes?
        \begin{itemize}
            \item From part (1), determine which structure gave rise to three nondegenerate stretching modes, and which gave rise to two.
        \end{itemize}
    \end{enumerate}
    \item This example shows that we can determine the stretching modes for just some functional groups.
    \item Be careful with what the basis set is!
    \item Example: Structures for \ce{OsO4N}.
    \begin{itemize}
        \item This molecule has four reasonable structures, having symmetry $C_{2v}$, $C_{3v}$, $C_{4v}$, and $C_s$. It is a great example! Especially when paired with preceding molecules using some subset of these character tables.
    \end{itemize}
\end{itemize}



\section{Raman Selection Rules and Normal Mode Analysis}
\begin{itemize}
    \item \marginnote{10/19:}PSet 2 is due at the beginning of class on Friday.
    \item Raman and IR are complementary, and together they can distinguish geometric possibilities of an unknown molecule.
    \item \textbf{Raman spectroscopy}: A type of spectroscopy which probes inelastic scattering of light where the loss in energy corresponds to a vibrational frequency (\textbf{Stokes shift}).
    \item In Raman, you go to \textbf{virtual energy states}.
    \item \textbf{Virtual energy state}: The coupling of a photon with a high energy state.
    \item \textbf{Rayleigh scattering}: The amount of energy put in is the amount of energy you get out. \emph{Also known as} \textbf{elastic scattering}.
    \begin{itemize}
        \item Doesn't give us a change, so we filter this out.
    \end{itemize}
    \item \textbf{Stokes Raman scattering}: The photon out has less energy than the photon in; some energy was scattered.
    \begin{itemize}
        \item We excite from the ground state to a virtual energy state, and then fall back down but not all the way to the ground state, i.e., the electron remains in an excited state even after emitting its photon.
    \end{itemize}
    \item \textbf{Anti-Stokes Raman scattering}: The photon out has more energy than the photon in; we start at a higher energy state and then fall back to ground.
    \begin{itemize}
        \item We excite from an excited state to a virtual energy state, and the fall back down all the way to the ground state.
    \end{itemize}
    \item A vibrational mode will be Raman active if the polarizability of the molecule changes during the vibration. A fundamental transition will be Raman active (i.e., will give rise to a Raman shift) if the normal mode involved belongs to the same representation as one or more of the components of the polarizability tensor of the molecule.
    \item Transition probability in Raman:
    \begin{equation*}
        M_w^2 = \left[ \int_{-\infty}^\infty\Psi_{v'}^*\hat{\alpha}\Phi_v\dd\tau \right]^2
    \end{equation*}
    \begin{itemize}
        \item Polarizability describes the shape of the electron cloud and can be described by a tensor.
        \begin{equation*}
            \begin{bmatrix}
                \mu_x\\
                \mu_y\\
                \mu_z\\
            \end{bmatrix}
            =
            \begin{bmatrix}
                \alpha_{xx} & \alpha_{xy} & \alpha_{xz}\\
                \alpha_{yx} & \alpha_{yy} & \alpha_{yz}\\
                \alpha_{zx} & \alpha_{zy} & \alpha_{zz}\\
            \end{bmatrix}
            \begin{bmatrix}
                E_x\\
                E_y\\
                E_z\\
            \end{bmatrix}
        \end{equation*}
        \item The matrix above is the polarizability tensor, and the right vector above describes the electric field.
        \item Three $\alpha$ values are redundant: The matrix above is symmetric, so
        \begin{align*}
            \alpha_{xz} &= \alpha_{zx}&
            \alpha_{yz} &= \alpha_{zy}&
            \alpha_{yx} &= \alpha_{xy}
        \end{align*}
        \item Thus, $\hat{\alpha}$ has six different components.
        \item If you do the math, you learn that the transition is allowed if the symmetric IRR is in the direct product and the normal mode is a quadratic function.
    \end{itemize}
    \item Recall: Linear for IR, quadratic for Raman.
    \item Example: The nine modes of vibration, two pairs of which are degenerate, are derived below for \ce{XeF4}. Show that the $b_{1g}$ fundamental transition is allowed for Raman but forbidden for infrared.
    \begin{equation*}
        \Gamma_\text{vibs} = a_{1g}+b_{1g}+b_{2g}+a_{2g}+a_{2u}+b_{2u}+2e_u
    \end{equation*}
    \begin{itemize}
        \item Checking the $D_{4h}$ character table, we see that the two linear representations are $a_{2u}$ and $e_u$. Thus,
        \begin{equation*}
            \Psi^*_{v'}\hat{\mu}\Psi_v \sim b_{1g}
            \begin{pmatrix}
                a_{2u}\\
                e_u\\
            \end{pmatrix}
            a_{1g} =
            \begin{pmatrix}
                b_{2u}\\
                e_u\\
            \end{pmatrix}
        \end{equation*}
        Thus, since $a_{1g}$ doesn't appear, the IR transition is not allowed.
        \item Checking the $D_{4h}$ character table, we see that the four quadratic representations are $a_{1g}$, $b_{1g}$, $b_{2g}$, and $e_g$. Thus,
        \begin{equation*}
            \Psi^*_{v'}\hat{\mu}\Psi_v \sim b_{1g}
            \begin{pmatrix}
                a_{1g}\\
                b_{1g}\\
                b_{2g}\\
                e_g\\
            \end{pmatrix}
            a_{1g} =
            \begin{pmatrix}
                b_{1g}\\
                a_{1g}\\
                a_{2g}\\
                e_g\\
            \end{pmatrix}
        \end{equation*}
        Thus, since $a_{1g}$ appears, the Raman transition is allowed.
    \end{itemize}
    \item \textbf{Fundamental transition}: A transition starting from the ground state.
    \begin{itemize}
        \item $\Psi_v$ is always the totally symmetric IRR for a fundamental transition.
        \item The wording "$b_{1g}$ fundamental transition" in the previous example means "the transition form $a_{1g}$ to $b_{1g}$."
        \item Essentially, we're not dealing with overtones, not starting from an excited state, no coupling, nothing fancy.
        \item We will briefly talk about overtones later.
    \end{itemize}
    \item Example: Determine the symmetries and activities of the normal modes of vibration for the cyclopropenyl cation. Use all atoms ($N=6$), i.e., use the Cartesion displacement method.
    \begin{itemize}
        \item Point group: $D_{3h}$.
        \item $\Gamma_{xyz}=(3,0,-1,1,-2,1)$.
        \item $\Gamma_\text{unmoved}=(6,0,2,6,0,2)$.
        \item $\Gamma_{3N}=(18,0,-2,6,0,2)=2A_1'+2A_2'+4E'+2A_2'+2E'$.
        \item $\Gamma_\text{vibs}=2A_1'+A_2'+3E'+A_2''+E''$.
        \item Operators:
        \begin{align*}
            \hat{\mu} &=
            \begin{pmatrix}
                e'\\
                a_2''\\
            \end{pmatrix}&
            \hat{\alpha} &=
            \begin{pmatrix}
                a_1'\\
                e'\\
                e''\\
            \end{pmatrix}
        \end{align*}
        \item $a_1'$ is Raman active, $e'$ is both, $a_2''$ is IR active, and $e''$ is both.
    \end{itemize}
    \item You only need uppercase Mulliken symbols for character tables and Tanabe-Sugano diagrams.
    \item \textbf{Rule of mutual exclusion}: No normal modes can be both infrared and Raman active in a molecule that possesses a center of symmetry.
    \item Question: Spectrum analysis. What symmetry element must be present?
    \begin{itemize}
        \item Observation: Lack of coincidental IR and Raman peaks in the spectra of benzene.
        \item Thus, no linear bases or quadratic bases overlap. The linear must all be $-1$ and the quadratic must all be 1.
        \item Thus, an inversion $i$ is present.
    \end{itemize}
\end{itemize}



\section{Special Spectroscopic Bands}
\begin{itemize}
    \item \marginnote{10/21:}Calculators that aren't connected to the internet are permitted. Arrive by 9:25. Think of it as a quiz more than an exam --- there's only so much you can do in 50 minutes.
    \item \textbf{Polarization}: When electric fields are restricted to a specific direction by filtration.
    \item Recall \ce{H2O} and its vibrational modes $\nu_1,\nu_2,\nu_3$, which are symmetric stretch, bending, and asymmetric stretch, respectively. Consider the \emph{hypothetical} case where we can "hold" water molecules such that they are oriented on the Cartesian plane.
    \begin{figure}[h!]
        \centering
        \begin{subfigure}[b]{0.49\linewidth}
            \centering
            \begin{tikzpicture}[
                every node/.style={black}
            ]
                \small
                \draw (0,2.1) -- node[rotate=90,above]{Transmission} (0,0) -- node[below]{\si{\per\centi\meter}} (5,0);
    
                \footnotesize
                \draw [rex,thick] (0.1,2)
                    -- ++(0.3,0)
                    to [out=0,in=180,out looseness=0.7,in looseness=0.5] ++(0.5,-1.5) node[below]{$b_2$}
                    to [out=0,in=180,out looseness=0.5,in looseness=0.7] ++(0.5,1.5)
                    -- ++(0.5,0)
                    to [out=0,in=180,out looseness=0.7,in looseness=0.5] ++(0.3,-0.8) node[below]{$a_1$}
                    to [out=0,in=180,out looseness=0.5,in looseness=0.7] ++(0.3,0.8)
                    -- ++(1,0)
                    to [out=0,in=180,out looseness=0.7,in looseness=0.5] ++(0.3,-0.8) node[below]{$a_1$}
                    to [out=0,in=180,out looseness=0.5,in looseness=0.7] ++(0.3,0.8)
                    -- ++(0.8,0)
                ;
            \end{tikzpicture}
            \caption{Unpolarized.}
            \label{fig:H2O-IRPola}
        \end{subfigure}
        \begin{subfigure}[b]{0.49\linewidth}
            \centering
            \begin{tikzpicture}[
                every node/.style={black}
            ]
                \small
                \draw (0,2.1) -- node[rotate=90,above]{Transmission} (0,0) -- node[below]{\si{\per\centi\meter}} (5,0);
    
                \footnotesize
                \draw [rex,thick] (0.1,2)
                    -- ++(0.3,0)
                    to [out=0,in=180,out looseness=0.7,in looseness=0.5] ++(0.5,0)
                    to [out=0,in=180,out looseness=0.5,in looseness=0.7] ++(0.5,0)
                    -- ++(0.5,0)
                    to [out=0,in=180,out looseness=0.7,in looseness=0.5] ++(0.3,-0.8) node[below]{$a_1$}
                    to [out=0,in=180,out looseness=0.5,in looseness=0.7] ++(0.3,0.8)
                    -- ++(1,0)
                    to [out=0,in=180,out looseness=0.7,in looseness=0.5] ++(0.3,-0.8) node[below]{$a_1$}
                    to [out=0,in=180,out looseness=0.5,in looseness=0.7] ++(0.3,0.8)
                    -- ++(0.8,0)
                ;
            \end{tikzpicture}
            \caption{$z$-polarized.}
            \label{fig:H2O-IRPolb}
        \end{subfigure}\\[2em]
        \begin{subfigure}[b]{0.49\linewidth}
            \centering
            \begin{tikzpicture}[
                every node/.style={black}
            ]
                \small
                \draw (0,2.1) -- node[rotate=90,above]{Transmission} (0,0) -- node[below]{\si{\per\centi\meter}} (5,0);
    
                \footnotesize
                \draw [rex,thick] (0.1,2)
                    -- ++(0.3,0)
                    to [out=0,in=180,out looseness=0.7,in looseness=0.5] ++(0.5,0)
                    to [out=0,in=180,out looseness=0.5,in looseness=0.7] ++(0.5,0)
                    -- ++(0.5,0)
                    to [out=0,in=180,out looseness=0.7,in looseness=0.5] ++(0.3,0)
                    to [out=0,in=180,out looseness=0.5,in looseness=0.7] ++(0.3,0)
                    -- ++(1,0)
                    to [out=0,in=180,out looseness=0.7,in looseness=0.5] ++(0.3,0)
                    to [out=0,in=180,out looseness=0.5,in looseness=0.7] ++(0.3,0)
                    -- ++(0.8,0)
                ;
            \end{tikzpicture}
            \caption{$x$-polarized.}
            \label{fig:H2O-IRPolc}
        \end{subfigure}
        \begin{subfigure}[b]{0.49\linewidth}
            \centering
            \begin{tikzpicture}[
                every node/.style={black}
            ]
                \small
                \draw (0,2.1) -- node[rotate=90,above]{Transmission} (0,0) -- node[below]{\si{\per\centi\meter}} (5,0);
    
                \footnotesize
                \draw [rex,thick] (0.1,2)
                    -- ++(0.3,0)
                    to [out=0,in=180,out looseness=0.7,in looseness=0.5] ++(0.5,-1.5) node[below]{$b_2$}
                    to [out=0,in=180,out looseness=0.5,in looseness=0.7] ++(0.5,1.5)
                    -- ++(0.5,0)
                    to [out=0,in=180,out looseness=0.7,in looseness=0.5] ++(0.3,0)
                    to [out=0,in=180,out looseness=0.5,in looseness=0.7] ++(0.3,0)
                    -- ++(1,0)
                    to [out=0,in=180,out looseness=0.7,in looseness=0.5] ++(0.3,0)
                    to [out=0,in=180,out looseness=0.5,in looseness=0.7] ++(0.3,0)
                    -- ++(0.8,0)
                ;
            \end{tikzpicture}
            \caption{$y$-polarized.}
            \label{fig:H2O-IRPold}
        \end{subfigure}
        \caption{Polarized IR spectra of \ce{H2O}.}
        \label{fig:H2O-IRPol}
    \end{figure}
    \begin{itemize}
        \item The unpolarized IR spectrum would be the result of what we predicted last time.
        \item The $z$-polarized spectrum: $\nu_1,\nu_2$ are $a_1$ and hence transform with the same symmetry as $z$. You filter out the $b_2$ for a $z$-polarized spectrum.
        \item $y$-polarized gives you just $b_2$.
        \item $x$-polarized gives you nothing.
    \end{itemize}
    \item Linear functions give you the right answer, but you can also rationalize from the vectors.
    \begin{itemize}
        \item For example, drawing out the vectors for $\nu_1$ in \ce{H2O}, you see that the major dipole moment is in the $z$-direction. Same for $\nu_2$. However, for $\nu_3$, the major dipole moment is in the $y$-direction.
        \item What we're polarizing here is the incoming IR radiation. If all molecules were held with the correct orientation and then we shot $z$-polarized light at them, only vibrations in the $z$ direction would get excited ($x,y$ are equally and oppositely cancelled).
    \end{itemize}
    \item Polarization is most commonly used in Raman spectroscopy.
    \item Experimental setup.
    \begin{figure}[h!]
        \centering
        \begin{subfigure}[b]{0.49\linewidth}
            \centering
            \includegraphics[width=0.8\linewidth]{../ExtFiles/ramanPola.png}
            \caption{Parallel scattering.}
            \label{fig:ramanPola}
        \end{subfigure}
        \begin{subfigure}[b]{0.49\linewidth}
            \centering
            \includegraphics[width=0.8\linewidth]{../ExtFiles/ramanPolb.png}
            \caption{Perpendicular scattering.}
            \label{fig:ramanPolb}
        \end{subfigure}
        \caption{Raman polarization setup.}
        \label{fig:ramanPol}
    \end{figure}
    \begin{itemize}
        \item We place the sample solution at the origin of our coordinate system, shoot $xz$-polarized light down one axis, and measure the polarization of the emitted radiation down a perpendicular axis.
        \item From this setup, we can determine the extent to which the molecules that absorb plane-polarized light emit light in the same plane or in a perpendicular plane.
    \end{itemize}
    \item \textbf{Depolarization ratio}: The following quantity, where $I_\perp$ is the intensity of the scattered light polarized in the plane perpendicular to the incident light and $I_\parallel$ is the intensity of the scattered light in the same plane as the incident light. \emph{Denoted by} $\bm{\rho}$. \emph{Given by}
    \begin{equation*}
        \rho = \frac{I_\perp}{I_\parallel}
    \end{equation*}
    \item Depolarized (unpolarized) bands will exhibit $\rho\approx 3/4$. In contrast, polarized bands will exhibit $\rho$ values between 0 and $3/4$.
    \begin{itemize}
        \item Polarized bands appear from vibrations that are totally symmetric.
        \item The more highly symmetric the molecule, the closer $\rho$ is to zero (polarization is high).
    \end{itemize}
    \item Example: For each IR transition of the cyclopropenyl cation, determine the direction of the adsorption. Determine which Raman active fundamentals of the compound are polarized or depolarized.
    \begin{itemize}
        \item IR: The polarization of the $e'$ adsorption is $(x,y)$ and the direction of polarization of $a_2''$ is $z$.
        \begin{itemize}
            \item What does the degenerate polarization of $e'$ mean?? Does it mean that either $x$- or $y$-polarized light will excite this transition?
        \end{itemize}
        \item Raman: $a_1'$ is your most polarized Raman active mode. $e'$ and $e''$ are polarized but not as much.
    \end{itemize}
    \item Most IR spectra show more bands than we predict for fundamentals. These are \textbf{overtones}, \textbf{combination bands}, and \textbf{hot bands}.
    \item \textbf{Overtone}: A band that occurs when a mode is excited beyond $v=1$ by a single photon.
    \emph{picture}
    \begin{itemize}
        \item Example: If our three normal modes are $\Psi_1(0)\Psi_2(0)\Psi_3(0)$ and we excite one of them such that it goes to a third state $\Psi_1(0)\Psi_2(3)\Psi_3(0)$. This is the \textbf{second overtone} of the normal mode $\nu_2$. We are taking $v=0$ to $v=3$ here.
        \item The energy of this transition would be approximately 3 times the fundamental $v_0\to v_1$.
    \end{itemize}
    \item \textbf{Combination band}: A band that occurs when more than one vibration is excited by one photon. \emph{Also known as} \textbf{combo band}.
    \begin{itemize}
        \item Example:
        \begin{equation*}
            \Psi_1(0)\Psi_2(0)\Psi_3(0) \to \Psi_1(1)\Psi_2(1)\Psi_3(0)
        \end{equation*}
        \item Example:
        \begin{equation*}
            \Psi_1(0)\Psi_2(0)\Psi_3(0) \to \Psi_1(2)\Psi_2(0)\Psi_3(1)
        \end{equation*}
        \begin{itemize}
            \item I.e., you can throw in overtones, too.
        \end{itemize}
        \item What this means for the energy: Energy of the combination band transition is the sum of the energies of the individual transitions.
        \begin{itemize}
            \item Energy for example 1: $v_1+v_2$.
            \item Energy for example 2: $2v_1+v_3$.
        \end{itemize}
    \end{itemize}
    \item \textbf{Hot band}: A band that occurs when an already excited vibration is further excited.
    \begin{itemize}
        \item Example:
        \begin{equation*}
            \Psi_1(0)\Psi_2(1)\Psi_3(0) \to \Psi_1(0)\Psi_2(2)\Psi_3(0)
        \end{equation*}
        \item The probability of this event depends on the temperature because it relies on the thermal population of an already excited state.
        \item The population increases as a function of temperature.
        \item Thermal population of the initial state is low, but it increases with temperature and hence is called a \emph{hot} band.
    \end{itemize}
    \item The selection rules for predicting whether an overtone, combo band, or hot band is possible use the same direct product math.
    \begin{itemize}
        \item We won't cover this in depth, though.
    \end{itemize}
    \item \textbf{Fermi resonance}: The mixing of two states, which can have two effects:
    \begin{enumerate}
        \item The overtone can gain intensity from the nearby fundamental of the same symmetry.
        \item Both energy levels are shifted away from each other.
    \end{enumerate}
    \item Predicting when Fermi resonance occurs is hard; it is done only after ruling out the possibility of an overtone, combo band, or hot band as accounting for your data.
    \item Example: \ce{CO2}.
    \begin{itemize}
        \item Since there are four normal modes, you might predict four IR bands.
        \item The $\delta_d$ bending modes have $E=\SI{667}{\per\centi\meter}$. You'd predict their first overtone to have $E=\SI{1334}{\per\centi\meter}$. However, this mode has a similar symmetry to the symmetric stretch $\nu_s$ at $E=\SI{1337}{\per\centi\meter}$. Thus, they will mix via Fermi resonance, producing two really strong bands at \SI{1388}{\per\centi\meter} and \SI{1286}{\per\centi\meter}.
        \item According to Wuttig, it is purely a coincidence that energies in this example are similar; states don't only mix when their energies are similar.
    \end{itemize}
\end{itemize}




\end{document}